% !TeX spellcheck = en_US
%%%%%%%%%%%%%%%%%%%%%%%%%%%%%%%%%%%%%%%%%
% Masters/Doctoral Thesis 
% LaTeX Template
% Version 2.5 (27/8/17)
%
% This template was downloaded from:
% http://www.LaTeXTemplates.com
%
% Version 2.x major modifications by:
% Vel (vel@latextemplates.com)
%
% This template is based on a template by:
% Steve Gunn (http://users.ecs.soton.ac.uk/srg/softwaretools/document/templates/)
% Sunil Patel (http://www.sunilpatel.co.uk/thesis-template/)
%
% Template license:
% CC BY-NC-SA 3.0 (http://creativecommons.org/licenses/by-nc-sa/3.0/)
%
%%%%%%%%%%%%%%%%%%%%%%%%%%%%%%%%%%%%%%%%%

%----------------------------------------------------------------------------------------
%	PACKAGES AND OTHER DOCUMENT CONFIGURATIONS
%----------------------------------------------------------------------------------------

\documentclass[
11pt, % The default document font size, options: 10pt, 11pt, 12pt
%oneside, % Two side (alternating margins) for binding by default, uncomment to switch to one side
english, % ngerman for German
onehalfspacing,
%singlespacing, % Single line spacing, alternatives: onehalfspacing or doublespacing
%draft, % Uncomment to enable draft mode (no pictures, no links, overfull hboxes indicated)
%nolistspacing, % If the document is onehalfspacing or doublespacing, uncomment this to set spacing in lists to single
%liststotoc, % Uncomment to add the list of figures/tables/etc to the table of contents
%toctotoc, % Uncomment to add the main table of contents to the table of contents
%parskip, % Uncomment to add space between paragraphs
%nohyperref, % Uncomment to not load the hyperref package
headsepline, % Uncomment to get a line under the header
%chapterinoneline, % Uncomment to place the chapter title next to the number on one line
%consistentlayout, % Uncomment to change the layout of the declaration, abstract and acknowledgements pages to match the default layout
]{theme} % The class file specifying the document structure

\usepackage[utf8]{inputenc} % Required for inputting international characters
\usepackage[T1]{fontenc} % Output font encoding for international characters

\usepackage{mathpazo} % Use the Palatino font by default

\usepackage[backend=bibtex, style=chem-acs, natbib=true]{biblatex}

\DeclareCiteCommand{\citeauthor}%
{\boolfalse{citetracker}%
	\boolfalse{pagetracker}%
	\usebibmacro{prenote}}
{\ifciteindex
	{\indexnames{labelname}}
	{}%
	\printtext[bibhyperref]{\printnames{labelname}}}
{\multicitedelim}
{\usebibmacro{postnote}}

\usepackage{subcaption}

\addbibresource{bibliography.bib} % The filename of the bibliography

\usepackage[autostyle=true]{csquotes} % Required to generate language-dependent quotes in the bibliography

\renewcommand{\thesection}{\arabic{section}}

\usepackage{multirow}
\usepackage{mhchem}
\usepackage{hyperref}
\usepackage{xcolor}
\usepackage{amsmath}

\renewcommand{\arraystretch}{1.5}
%----------------------------------------------------------------------------------------
%	MARGIN SETTINGS
%----------------------------------------------------------------------------------------

\geometry{
	paper=letterpaper, % Change to letterpaper for US letter
	inner=2.5cm, % Inner margin
	outer=2.5cm, % Outer margin
	bindingoffset=.5cm, % Binding offset
	top=1.5cm, % Top margin
	bottom=1.5cm, % Bottom margin
	%showframe, % Uncomment to show how the type block is set on the page
	}
\supervisor{Jaime \textsc{Forero}, Ph.D.} % Your supervisor's name, this is used in the title page, print it elsewhere with \supname
\examiner{} % Your examiner's name, this is not currently used anywhere in the template, print it elsewhere with \examname
\degree{Monograph} % Your degree name, this is used in the title page and abstract, print it elsewhere with \degreename

%----------------------------------------------------------------------------------------
%	THESIS INFORMATION
%----------------------------------------------------------------------------------------

\thesistitle{Orbits of black holes in galactic triaxial potentials} % Your thesis title, this is used in the title and abstract, print it elsewhere with \ttitle

\author{Juan S. \textsc{Barbosa Coy}} % Your name, this is used in the title page and abstract, print it elsewhere with \authorname
\addresses{} % Your address, this is not currently used anywhere in the template, print it elsewhere with \addressname

\subject{Physics} % Your subject area, this is not currently used anywhere in the template, print it elsewhere with \subjectname
\keywords{} % Keywords for your thesis, this is not currently used anywhere in the template, print it elsewhere with \keywordnames
\university{\href{http://www.uniandes.edu.co}{Universidad de los Andes}} % Your university's name and URL, this is used in the title page and abstract, print it elsewhere with \univname
\department{\href{http://fisica.uniandes.edu.co}{Departamento de F\'isica\\Facultad de Ciencias\\Universidad de los Andes}} % Your department's name and URL, this is used in the title page and abstract, print it elsewhere with \deptname
\group{\href{https://github.com/astroandes}{Astroandes}} % Your research group's name and URL, this is used in the title page, print it elsewhere with \groupname
\faculty{\href{http://ciencias.uniandes.com}{Facultad de Ciencias}} % Your faculty's name and URL, this is used in the title page and abstract, print it elsewhere with \facname

\AtBeginDocument{
	\hypersetup{pdftitle=\ttitle} % Set the PDF's title to your title
	\hypersetup{pdfauthor=\authorname} % Set the PDF's author to your name
	\hypersetup{pdfkeywords=\keywordnames} % Set the PDF's keywords to your keywords
}

\begin{document}

\frontmatter % Use roman page numbering style (i, ii, iii, iv...) for the pre-content pages

\pagestyle{plain} % Default to the plain heading style until the thesis style is called for the body content

%----------------------------------------------------------------------------------------
%	TITLE PAGE
%----------------------------------------------------------------------------------------

\begin{titlepage}
\begin{center}

\vspace*{.06\textheight}
{\scshape\LARGE \univname\par}\vspace{1.5cm} % University name
\textsc{\Large Monograph}\\[0.5cm] % Thesis type

\HRule \\[0.4cm] % Horizontal line
{\LARGE \bfseries \ttitle\par}\vspace{0.4cm} % Thesis title
\HRule \\[1.5cm] % Horizontal line
 
\begin{minipage}[t]{0.4\textwidth}
\begin{flushleft} \large
\emph{Author:}\\
\href{https://www.github.com/jsbarbosa}{\authorname} % Author name - remove the \href bracket to remove the link
\end{flushleft}
\end{minipage}
\begin{minipage}[t]{0.4\textwidth}
\begin{flushright} \large
\emph{Advisor:} \\
\href{http://wwwprof.uniandes.edu.co/~je.forero/}{\supname} % Supervisor name - remove the \href bracket to remove the link  
\end{flushright}
\end{minipage}\\[3cm]
 
\vfill

\large \textit{\degreename{} presented for the degree of physicist}\\[0.3cm] % University requirement text
%\textit{en el}\\[0.4cm]
\deptname\\[2cm] % Research group name and department name
%\groupname
Bogot\'a, Colombia

{\large \today}\\[4cm] % Date
%\includegraphics{Logo} % University/department logo - uncomment to place it
 
\vfill
\end{center}
\end{titlepage}

%----------------------------------------------------------------------------------------
%	DECLARATION PAGE
%----------------------------------------------------------------------------------------

%\begin{declaration}
%\addchaptertocentry{\authorshipname} % Add the declaration to the table of contents
%\noindent I, \authorname, declare that this thesis titled, \enquote{\ttitle} and the work presented in it are my own. I confirm that:
%
%\begin{itemize} 
%\item This work was done wholly or mainly while in candidature for a research degree at this University.
%\item Where any part of this thesis has previously been submitted for a degree or any other qualification at this University or any other institution, this has been clearly stated.
%\item Where I have consulted the published work of others, this is always clearly attributed.
%\item Where I have quoted from the work of others, the source is always given. With the exception of such quotations, this thesis is entirely my own work.
%\item I have acknowledged all main sources of help.
%\item Where the thesis is based on work done by myself jointly with others, I have made clear exactly what was done by others and what I have contributed myself.\\
%\end{itemize}
% 
%\noindent Signed:\\
%\rule[0.5em]{25em}{0.5pt} % This prints a line for the signature
% 
%\noindent Date:\\
%\rule[0.5em]{25em}{0.5pt} % This prints a line to write the date
%\end{declaration}

\cleardoublepage

%----------------------------------------------------------------------------------------
%	QUOTATION PAGE
%----------------------------------------------------------------------------------------

%\vspace*{0.2\textheight}
%
%\noindent\enquote{\itshape Thanks to my solid academic training, today I can write hundreds of words on virtually any topic without possessing a shred of information, which is how I got a good job in journalism.}\bigbreak
%
%\hfill Dave Barry

%----------------------------------------------------------------------------------------
%	ABSTRACT PAGE
%----------------------------------------------------------------------------------------

\begin{abstract}
	\addchaptertocentry{\abstractname} % Add the abstract to the table of contents
	Observations have confirmed that black hole mergers can emit gravitational waves. For such cases, conservation of momentum dictates that the merged system must move away from its initial position, producing what is known as a black hole kick. The black hole can be kicked with velocities up to 3000 km/s, depending on the mass ratio and spin alignment of the bodies. The behaviour of the kicked black hole depends on the gravitational potential provided by the stars, gas and dark matter of the host galaxy. In this work we used triaxial galactic potentials to follow the fate of such black holes. Our model takes into account the effects of dynamical friction and accretion. We will show how the settling times (the time it takes to go back at the potential minimum after the kick) depend on the stellar mass density and what is the influence of the potential triaxiallity to induce stochasticity on the orbits. We finalize by commenting how these results can impact abundance estimates of active galactic nuclei across cosmic times.
	
\end{abstract}

\begin{resumen}
	\addchaptertocentry{Resumen} % Add the abstract to the table of contents
	Las observaciones han confirmado que las fusiones de agujeros negros pueden emitir ondas gravitacionales. Para tales casos, la conservación del impulso dicta que el sistema fusionado debe alejarse de su posición inicial, produciendo lo que se conoce como una patada de agujero negro. El agujero negro puede experimentar un \textit{kick} con velocidades de hasta 3000 km/s, dependiendo de la proporción de masa y la alineación de espines de los cuerpos. El comportamiento del agujero negro depende del potencial gravitatorio provisto por las estrellas, el gas y la materia oscura de la galaxia huésped. En este trabajo, utilizamos potenciales galácticos triaxiales para seguir el destino de estos agujeros negros. Nuestro modelo tiene en cuenta los efectos de la fricción dinámica y la accreción del agujero negro. Mostraremos cómo los tiempos de asentamiento (el tiempo que se tarda en volver al mínimo potencial) dependen de la densidad de masa estelar y de la influencia de la triaxialidad del potencial, la cual induce estocasticidad en las órbitas. Finalizamos comentando cómo estos resultados pueden impactar las estimaciones de abundancia de núcleos galácticos activos a través de los tiempos cósmicos.
\end{resumen}


%----------------------------------------------------------------------------------------
%	ACKNOWLEDGEMENTS
%----------------------------------------------------------------------------------------

%\begin{acknowledgements}
%\addchaptertocentry{\acknowledgementname} % Add the acknowledgements to the table of contents
%The acknowledgments and the people to thank go here, don't forget to include your project advisor\ldots
%\end{acknowledgements}

%----------------------------------------------------------------------------------------
%	LIST OF CONTENTS/FIGURES/TABLES PAGES
%----------------------------------------------------------------------------------------
%
\tableofcontents % Prints the main table of contents
%
\addchaptertocentry{List of Figures}
\listoffigures % Prints the list of figures
%
\addchaptertocentry{List of Tables}
\listoftables % Prints the list of tables

%----------------------------------------------------------------------------------------
%	ABBREVIATIONS
%----------------------------------------------------------------------------------------
%
%\begin{abbreviations}{ll} % Include a list of abbreviations (a table of two columns)
%
%\textbf{LAH} & \textbf{L}ist \textbf{A}bbreviations \textbf{H}ere\\
%\textbf{WSF} & \textbf{W}hat (it) \textbf{S}tands \textbf{F}or\\
%
%\end{abbreviations}

%----------------------------------------------------------------------------------------
%	PHYSICAL CONSTANTS/OTHER DEFINITIONS
%----------------------------------------------------------------------------------------
%
%\begin{constants}{lr@{${}={}$}l} % The list of physical constants is a three column table
%
%% The \SI{}{} command is provided by the siunitx package, see its documentation for instructions on how to use it
%
%Speed of Light & $c_{0}$ & \SI{2.99792458e8}{\meter\per\second} (exact)\\
%%Constant Name & $Symbol$ & $Constant Value$ with units\\
%
%\end{constants}

%----------------------------------------------------------------------------------------
%	SYMBOLS
%----------------------------------------------------------------------------------------

%\begin{symbols}{lll} % Include a list of Symbols (a three column table)
%
%$a$ & distance & \si{\meter} \\
%$P$ & power & \si{\watt} (\si{\joule\per\second}) \\
%%Symbol & Name & Unit \\
%
%\addlinespace % Gap to separate the Roman symbols from the Greek
%
%$\omega$ & angular frequency & \si{\radian} \\
%
%\end{symbols}

%----------------------------------------------------------------------------------------
%	DEDICATION
%----------------------------------------------------------------------------------------

%\dedicatory{For/Dedicated to/To my\ldots} 

%----------------------------------------------------------------------------------------
%	THESIS CONTENT - CHAPTERS
%----------------------------------------------------------------------------------------

\mainmatter % Begin numeric (1,2,3...) page numbering

\pagestyle{thesis} % Return the page headers back to the "thesis" style

% Include the chapters of the thesis as separate files from the Chapters folder
% Uncomment the lines as you write the chapters

\newcommand{\keyword}[1]{\textit{#1}}
\newcommand{\sm}[0]{$M_\odot$}
\newcommand{\todo}[1]{\texttt{\color{red}\#TODO: #1}}
\newcommand{\erf}[1]{\text{erf}\left(#1\right)}

%\setstretch{2.0}
% !TeX spellcheck = es_ANY
% Chapter 1

%\chapter{Chapter Title Here} % Main chapter title
%
%\label{Chapter1} % For referencing the chapter elsewhere, use \ref{Chapter1} 

%----------------------------------------------------------------------------------------

% Define some commands to keep the formatting separated from the content 
\newcommand{\keyword}[1]{\textit{#1}}
%\newcommand{\tabhead}[1]{\textbf{#1}}
%\newcommand{\code}[1]{\texttt{#1}}
%\newcommand{\file}[1]{\texttt{\bfseries#1}}
%\newcommand{\option}[1]{\texttt{\itshape#1}}

%----------------------------------------------------------------------------------------

\section{Introducción}
	La Teor\'ia de la Relatividad General de Albert Einstein fue publicada en 1915, de dicha teor\'ia surgen las ondas gravitacionales como una, entre varias prediciones, las cuales incluyen lentes gravitacionales en donde cuerpos masivos modifican la trayectoria de la luz, as\'i como la dilataci\'on del tiempo. El t\'ermino ``ondas gravitacionales'' fue introduccido por primera vez en una publicaci\'on de Henri Poincaré, 10 a\~nos antes, donde propon\'ia la primera ecuaci\'on para un campo gravitacional invariante ante transformaciones de Lorentz \cite{straumann2012general, bassan2014advanced}. Actualmente se entiende por ondas gravitacionales a las variaciones periódicas de la geometría del espacio-tiempo, y tienen su origen en que la energía y densidad de momento de un campo gravitacional actúan a su vez como fuentes de gravedad \cite{hoyng2006gravitational}. A pesar que han pasado más de 100 años desde la publicación de la teoría, aun hoy en día existen vacíos en el entendimiento y las implicaciones de las ecuaciones de Einstein. Lo anterior se debe en parte a la dificultad de resolver las ecuaciones para situaciones físicas de interés, por ejemplo en el caso de las ondas gravitacionales sólo se pueden resolver analíticamente para campos débiles usando una forma lineal de estas ecuaciones. Los cuales detectaron en 2016 de forma simultánea y por primera vez en la historia de la humanidad una onda gravitacional, lo cual fue reconocido por la comunidad científica en 2017 con el Premio Nobel de Física \cite{brugmann2018fundamentals}.  
	
	Si bien cuales quiera dos objetos con masas desiguales generan ondas gravitacionales al orbitar sobre el centro de masa, la mayor\'ia de campos gravitacionales en el universo son d\'ebiles, entre las pocas escepciones se encuentran los campos cercanos a cuerpos extremadamente masivos, como agujeros negros y estrellas de neutrones, raz\'on por la cual	en la pr\'actica solo las ondas gravitacionales provenientes de cuerpos extremadamente masivos son detectables \cite{straumann2012general}. Esto restringe las fuentes a sistemas binarios de estrellas, agujeros negros y supernovas (siempre que la explosi\'on no sea sim\'etrica). En el caso de los agujeros negros las ondas gravitacionales constituyen una herramienta muy poderosa para su estudio, pues a excepción de su gravedad, y su tamaño, un agujero negro es muy similar a cualquier otro objeto en el universo, bien sea una estrella o un planeta \cite{meier2012black}. Dada la magnitud de la gravedad generada, no es posible que la luz escape de él. Esto da lugar a una superficie invisible, en el caso de encontrarse cerca, es posible detectar un agujero negro pues el paso de este por el firmamento ocultar\'ia las estrellas del fondo. Pero a largas distancias, este efecto es poco apreciable y por ende no es posible observarlos por un instrumento \'optico como un telescopio. Por otro lado, cuando la distancia entre el observador y un agujero negro es muy grande, sus efectos gravitacionales son poco distintos a los presentados por un objeto con la misma masa, pero un volumen considerablemente mayor, pues a largas distancias ambas serán aproximadamente masas puntuales \cite{meier2012black}.
	
	Los esfuerzos por detectar ondas gravitacionales empezaron con el uso de barras resonantes por el f\'isico estadounidense Joe Weber en 1965 \cite{weber1967gravitational, bassan2014advanced}. Por varias d\'ecadas, la tecnolog\'ia, la sensibilidad y la extensi\'on de las barras de Weber mejor\'o, a tal punto de formar la primera red de observaci\'on global ondas gravitacionales. Sin embargo el acoplamiento entre materia y ondas es tan peque\~no, que en el caso de la detecci\'on de ondas gravitacionales existen factores que en otros contextos no seri\'an tan relevantes, como es el caso de las fuentes de ruido, entre los cuales se encuentran: t\'ermico y s\'ismico, \textit{shot noise} y su frecuencia \cite{bassan2014advanced}. Si bien los detectores de interferometr\'ia no estaban excentos de estas fuentes de ruido, presentan mayor sensitividad que las barras resonantes lo cual los hace los intrumentos m\'as usados hoy en d\'ia, ejemplos de estos son: LIGO (\textit{Laser Interferometer Gravitational-Wave Observatory} por sus siglas en ingl\'es) y Virgo, en Estados Unidos y Europa, correspondientemente \cite{abbott2009ligo, acernese2008status, bassan2014advanced}.
	
	\begin{figure}[h]
		\centering
		\includegraphics[width=0.65\linewidth]{Figures/binarySystem}
		\caption{Sistema binario con cuerpos de masas $m_1$ y $m_2$ ($m_2 > m_1$), emisor de ondas gravitacionales \cite{hughes2005black}.}
		\label{fig: binary}
	\end{figure}
	
	En el caso de un sistema binario como el de la \autoref{fig: binary}, visto desde la gravitación universal de Newton, se tendría como solución a las ecuaciones de movimiento trayectorias elípticas en concordancia con las leyes de Kepler. Sin embargo, al considerar la Relatividad General, se producirán ondas que llevan energ\'ia y momento \cite{hughes2005black, hoyng2006gravitational, brugmann2018fundamentals}. Esto ocasiona que poco a poco la \'orbita decaiga y los dos objetos se fusionen en un \'unico cuerpo. En el momento en que tiene lugar la fusi\'on de los cuerpos la amplitud de la onda aumenta considerablemente. Un fen\'omeno que se ha comprendido durante bastante tiempo, sin embargo sus implicaciones han sido poco estudiadas, son aquellas que se relacionan con el momento lineal que lleva la onda. Esto implica que al aumentar la amplitud de la onda con la fusi\'on, tambi\'en lo hace el momento lineal de la onda ($\mathbf{P}_{\text{ejected}}$), por lo cual la fusi\'on ocasiona que el centro de masa se mueva en direcci\'on opuesta con momento $\mathbf{P}_{\text{recoil}}$ para conservar el momento del sistema. A este movimiento del centro de masa se le conoce con el nombre de retroceso o patada (\textit{recoil} o \textit{kick}) \cite{hughes2005black} y fue descrito por Bonnor y Rotenberg en 1966 \cite{bonnor1966gravitational}, siendo un proceso que ocurre en una etapa con el nombre de coalescencia.
	
	En la mayoría de los casos el decaimíento de un sistema binario es tan lento que sucede en escalas de millones o billones de años, pero para el caso de los agujeros negros y estrellas de neutrones las ondas gravitacionales son de vital importancia. En estos mismos casos los campos gravitacionales son fuertes, por lo cual no es posible usar las soluciones analíticas a las ecuaciones de Einstein, por lo cual los métodos numéricos adquieren particular relevancia, pues aplicados a simulaciones computacionales permiten obtener soluciones a las ecuaciones de Einstein y de esta forma proveer un marco de referencia teórico para el estudio de agujeros negros binarios, estrellas de neutrones y ondas gravitacionales en general \cite{brugmann2018fundamentals}. Aun cuando los m\'etodos num\'ericos constituyen la mejor forma de estudio de estos sistemas est\'an lejos de ser perfectos pues incluso peque\~nos errores de aproximaci\'on num\'erica pueden dar lugar a divergencias que ocasionan que la simulaci\'on falle \cite{brugmann2018fundamentals}. Por esta raz\'on se busca comparar la evoluci\'on de un sistema binario de dos agujeros negros de Schwarzchild (no contienen carga, ni rotan sobre su eje) en la etapa de coalescencia, usando distintos algoritmos de integraci\'on para las ecuaciones de movimiento de los cuerpos, as\'i como su influencia sobre el \textit{kick} generado.
%	Una forma de entender c\'omo sucede este fen\'omeno, es usando un sistema con masas desiguales. Se puede considerar dos cuerpos, uno con masa $m_1$ y otro con masa $m_2$, donde $m_2 > m_1$. En el caso cl\'asico, donde no existe ning\'un efecto nuevo, se tiene que el centro de masa orbitar\'ia alrededor de un c\'irculo. Sin embargo, cuando se tiene una onda que lleva energ\'ia y momento angular, se tiene que los cuerpos seguir\'ian una trayector\'ia en espiral, por lo cual al considerar toda una \'orbita ($\mathcal{O}$) se tendr\'ia que:
%	
%	\begin{equation}
%		\int\limits_{\mathcal{O}_1} m_1\vec{v}_1dl \neq \int\limits_{\mathcal{O}_2} m_2\vec{v}_2dl 
%	\end{equation}	5
	
\section{Objetivo general}
	Comparar la evoluci\'on de un sistema binario de dos agujeros negros de Schwarzchild en la etapa de coalescencia, usando distintos algoritmos de integraci\'on para las ecuaciones de movimiento de los cuerpos, as\'i como su influencia sobre el \textit{kick} generado.
		
\section{Objetivos específicos}
	\begin{itemize}
		\item Obtener las ecuaciones de movimiento para un sistema de dos cuerpos, usando las ecuaciones de Einstein.
		\item Realizar el cableado y conexiones electrónicas pertinentes al mismo.
		\item Calibración eléctrica, determinación de las señales de entrada y salida, flujo de las bombas hidráulicas y temperatura del baño.
		\item Calibración química, determinación de la entalpía molar, energía libre de Gibbs, entropía, y constante de equilibrio, del acomplejamiento del catión bario con éter 18-corona-6.
	\end{itemize}
	
\section{Metodología}
	Las simulaciones ser\'an realizadas en el cluster de la universidad (HPC) con el fin de paralelizar procesos, de forma que a haciendo uso de sus recursos se minimice el tiempo de simulación a fin de lograr la mayor cantidad de simulaciones posibles con el fin de obtener una cantidad significativa de datos de validación.
	
\section{Consideraciones éticas}
	Se manejará un repositorio de uso privado a través de Github en donde se encontrar\'an los códigos implementados en cada parte del proceso, junto con los resultados obtenidos, de forma que se asegure la reproducibilidad del modelo hallado, al mismo tiempo que se permita el seguimiento del uso de recursos. Adem\'as al mantener la informaci\'on abierta se asegura que no se están utilizando resultados obtenidos por otros investigadores de forma directa.
	
\section{Cronograma}
	\begin{table}[h]
		\centering
		\caption{Cronograma de actividades}
		\label{tb: cronograma}
		\footnotesize
		\begin{tabular}{|c|c|c|c|c|c|c|c|c|c|c|c|c|c|c|c|c|}
			\hline
			\rowcolor[HTML]{C0C0C0} 
			\cellcolor[HTML]{C0C0C0}                                       & \multicolumn{16}{c|}{\cellcolor[HTML]{C0C0C0}\textbf{Semana}} \\ \cline{2-17} 
			\rowcolor[HTML]{EFEFEF} 
			\multirow{-2}{*}{\cellcolor[HTML]{C0C0C0}\textbf{Actividades}} & \textbf{1} & \textbf{2} & \textbf{3} & \textbf{4} & \textbf{5} & \textbf{6} & \textbf{7} & \textbf{8} & \textbf{9} & \textbf{10} & \textbf{11} & \textbf{12} & \textbf{13} & \textbf{14} & \textbf{15} & \textbf{16} \\ \hline
			\cellcolor[HTML]{EFEFEF}
			\textbf{Revisión bibliográfica} & x & x & x & & & & x & x & x & x & & & & x & x & x \\ \hline
			\cellcolor[HTML]{EFEFEF}\textbf{Revisión de manuales} & x & x & x & x & & & & & & & & & & & & \\ \hline
			\cellcolor[HTML]{EFEFEF}\textbf{Ensamble del equipo} & x & x & x & x & & & & & & & & & & & & \\ \hline
			\cellcolor[HTML]{EFEFEF}\textbf{Cableado y electrónica} & & & x & x & x & x & & & & & & & & & & \\ \hline
			\cellcolor[HTML]{EFEFEF}\textbf{Calibración eléctrica} & & & & & & x & x & x & & & & & & & & \\ \hline
			\cellcolor[HTML]{EFEFEF}\textbf{Calibración química} & & & & & & & & & x & x & x & x & & & & \\ \hline
			\cellcolor[HTML]{EFEFEF}\textbf{Análisis de datos} & & & & & & & & & & & x & x & x & & & \\ \hline
			\cellcolor[HTML]{EFEFEF}\textbf{Elaboración del documento} & & & & & & & x & x & x & x & & & & x & x & x \\ \hline
			\cellcolor[HTML]{EFEFEF}\textbf{Presentación del proyecto} & & & & & & & & & & & & & & & & x \\ \hline
		\end{tabular}
	\end{table}
% !TeX spellcheck = en_US
% Chapter 1

%\chapter{Chapter Title Here} % Main chapter title
%
%\label{Chapter1} % For referencing the chapter elsewhere, use \ref{Chapter1} 

%----------------------------------------------------------------------------------------

% Define some commands to keep the formatting separated from the content 

%\newcommand{\option}[1]{\texttt{\itshape#1}}

%----------------------------------------------------------------------------------------

\chapter{Methodology}
	Some of the simulation parameters are dependent of the cosmological model used, unless otherwise specified, all data is acquired using the $\Lambda$-CDM model with a matter density parameter $\Omega_M = 0.309$, $\Omega_\Lambda = 0.6911$, and a baryonic fraction $f_b = 0.156$ \cite{choksi2017recoiling}. 
	
	\section{Units}
		Computer simulations are sensitive to rounding errors due to the lack of infinite precision when representing decimal numbers. Really small numbers as well as really big ones tend to have bigger errors than those close to the unity, as can be seen on \autoref{fig: IEEE-754}.
		\begin{figure}[h]
			\centering
			\includegraphics[width=0.8\linewidth]{"../Files/Week 3/floating"}
			\caption{Floating point precision for different values, for a 32 bit and 64 bit holders.}
			\label{fig: IEEE-754}
		\end{figure}
		
		Under the International System of Units, distances are measured on meters, times on seconds, and masses on kilograms, nevertheless black holes are too heavy to be measured on kilograms, galaxies sizes too big to be quantified on meters, and time scales too large for a second. Because of that, the following units will be used throughout this document:
		\begin{table}[h]
			\centering
			\caption{Units of measure used on the simulations.}
			\label{tb: units}
			\begin{tabular}{c|c}
				\hline
				\textbf{Physical property} & \textbf{unit} \\
				\hline
				Length & 1 kilo-parsec (kpc) \\
				Mass & $10^5$ solar masses ($10^5$ \sm) \\
				Time & 1 giga-year (Gyr) \\
				\hline
			\end{tabular}
		\end{table}
	
		Along with the change of units, the universal gravitational constant and the Hubble parameter values are required to change.
		
		\subsection{Universal gravitational constant}
			First quantified by Henry Cavendish the gravitational constant has a value of $G_0 = 6.67408\times10^{-11}$ on SI units of m$^3$s$^{-2}$kg$^{-1}$. With the units of length, mass and time on \autoref{tb: units}, the constant of gravity used is given by:
			\begin{equation}
				\begin{array}{ccl}
					G & = & G_0 \left(\dfrac{1 \text{ kpc}^3}{\left(3.0857\times10^{19}\right)^3  \text{ m}^3}\right)\left(\dfrac{\left(3.154\times10^{16}\right)^2 \text{ s}^2}{1 \text{ Gyr}^2}\right)\left(\dfrac{1.98847\times10^{35} \text{ kg}}{10^5 M_\theta}\right) \\
					& = & 0.4493 \quad \dfrac{\text{kpc$^3$}}{\text{Gy$r^210^5$\sm}}	
				\end{array}
			\end{equation}
				
		
		\subsection{Hubble parameter}
			The Hubble constant is frequently used as $H_0 = 67.66 \pm 0.42$ kms$^{-1}$Mpc$^{-1}$ \cite{aghanim2018planck}, stating the speed of an astronomical body on kms$^{-1}$ at a distance of 1 Mpc. Nevertheless, the hubble constant has units of 1/time, thus, taking into account the units on \autoref{tb: units} one gets:
			\begin{equation}
				\begin{array}{ccl}
					H & = & H_0 \left(\dfrac{1 \text{ kpc}}{3.0857\times10^{16} \text{ km}}\right)\left(\dfrac{3.154\times10^{16} \text{ s}}{1 \text{ Gyr}}\right)\left(\dfrac{1 \text{ Mpc}}{1000 \text{ kpc}}\right) \\
					& \approx & 1.023 H_0 \times10^{-3} \text{ Gyr$^{-1}$} \\ 
					& = & 6.916\times10^{-2}\text{ Gyr$^{-1}$}
				\end{array}
			\end{equation}
			
			Although the Hubble parameter is often called Hubble constant, its value changes with time as can be seen on \autoref{fig: hubbleTime}. %In particular, at $z = 20$, the moment at which the kick occurs $H$ has a value of 3.699 Gyr$^{-1}$.
			\begin{figure}[h]
				\centering
				\includegraphics[width=0.8\linewidth]{"../Files/Week 5/hubble_time"}
				\caption{Dependency of the Hubble parameter with redshift.}
				\label{fig: hubbleTime}
			\end{figure}
		
	\section{Critical density and Virial Radius}
		Mass distributions used for the simulation of the host galaxy, are divergent for distances up to infinity. Because of this, the cumulative mass of all bodies within a given distance is called the virial mass and its value is taken as the mass of the whole system. The distance taken to calculate the virial mass is called virial radius ($R_\text{vir}$), and it is defined as the distance at which the average density of the galaxy is 200 times the critical density of the universe ($\rho_\text{crit}$).
		\begin{equation}\label{eq: critical_density}
			\rho_\text{crit} = \dfrac{3H(t)^2}{8\pi G}
		\end{equation}
		
		\begin{equation}\label{eq: R_vir_def}
			\begin{array}{c}
				\dfrac{M(R_\text{vir})}{V(R_\text{vir})} = \bar{\rho}(R_\text{vir}) =  200 \rho_\text{crit} = 75\dfrac{H(t)^2}{\pi G}\\
				\text{where $M(R_\text{vir})$ is the cumulative mass, and $V(R_\text{vir})$: the volume}
			\end{array}			
		\end{equation}
		
		The relation on \autoref{eq: critical_density} is found by considering the case where the geometry of the universe is flat, as a consequence it is said that the critical density is the minimum density required to stop the expansion of the universe \cite{binney2011galactic}.
		
	\section{Equation of motion}
		Trajectories of the kicked black holes were obtained by numerically solving the equation of motion on \autoref{eq: equationMotion}, where the first term on the right side of the equation is acceleration due to gravity, the second accounts for the drag of dynamical friction, while the third one is the deaceleration due to mass accretion of the black hole \cite{tanaka2009assembly, choksi2017recoiling}.
		\begin{equation}\label{eq: equationMotion}
			\ddot{\vec{x}} = a_\text{grav}(\vec{x})\hat{x} + \left(a_\text{DF}(\vec{x}, \dot{\vec{x}})-\dot{x}\dfrac{\dot{M_\bullet}(x, \dot{x})}{M_\bullet}\right)\dot{\hat{x}} \qquad \text{where $M_\bullet$ is the black hole mass}
		\end{equation}
		
		\subsection{Dynamical friction}
			As the black hole travels through the galaxy, dark matter, stars and gaseous materials from the medium interact with the black hole adding a drag force due to friction. Drag force is different in nature depending on its source, collisionless components, such as dark matter and stars, apply a drag force to the black hole that follows the standard Chandrasekhar formula \cite{binney2011galactic, madau2004effect, tanaka2009assembly, choksi2017recoiling}.
			\begin{equation}\label{eq: df_cl}
				a_\text{DF}^\text{cl}(\vec{x}, \dot{\vec{x}}) = -\dfrac{4\pi G^2}{\dot{x}^2} M_\bullet\rho(\vec{x})\ln\Lambda\left(\erf{X} - \dfrac{2}{\sqrt{\pi}}Xe^{-X^2}\right)\text{, } \quad \rho(\vec{x}) = \rho_\text{DM}(\vec{x}) + \rho_\text{stars}(\vec{x})
			\end{equation}
			\begin{equation}
				X \equiv \dfrac{|\dot{x}|}{\sqrt{2}\sigma_\text{DM}} \qquad \text{with } \sigma_\text{DM} = \sqrt{\dfrac{GM_\text{DM}}{2R_\text{vir}}}
			\end{equation}
			
			$\sigma_\text{DM}$ is called the local velocity dispersion of the dark matter halo, and since varies little over the entire host, can be taken as constant \cite{tanaka2009assembly, choksi2017recoiling}. The Coulomb logarithm ($\ln\Lambda$) is not known but authors take it in the range of 2 - 4 \cite{choksi2017recoiling}. Gas on the other hand is collisional, special care must be taken since gas can cool behind a passing object, such as a black hole \cite{choksi2017recoiling}. A hybrid model for the drag force was proposed by \citeauthor{tanaka2009assembly}, in which both subsonic and supersonic velocities are possible. To do so, a mach number was defined as:
			\begin{equation}
				\mathcal{M}(\dot{x}) \equiv \dfrac{|\dot{x}|}{c_s}
			\end{equation}
			
			where $c_s$ is the local sound speed, which depends on local temperature. It was found that temperature inside the halo varies less than a factor of 3, thus on the simulation it is assumed that the entire halo is isothermal at the virial temperature ($T_\text{vir}$) \cite{choksi2017recoiling}. The isothermal sound speed is \cite{barkana2001beginning}:
			\begin{equation}\label{eq: soundSpeed}
				c_s = \sqrt{\dfrac{\gamma R}{\mathcal{M}_w}T_\text{vir}} = \sqrt{\dfrac{\gamma R}{\mathcal{M}_w}\left(\dfrac{\mu m_p G M_h}{2k_BR_\text{vir}}\right)} = \sqrt{\dfrac{\gamma R\mu m_pG}{2\mathcal{M}_wk_B}} \sqrt{\dfrac{M_h}{R_\text{vir}}} \approx 0.614 \sqrt{\dfrac{M_h}{R_\text{vir}}}\text{ kpcGyr$^{-1}$}
			\end{equation}
			
			where $\mu$ is the value of the mean molecular weight of the gas ($\mathcal{M}_w$), $m_p$ is the proton mass and $\gamma$ is the adiabatic index \cite{barkana2001beginning}. Approximating the gas to a monoatomic one $\gamma \approx 5/3$, yields the last expression on \autoref{eq: soundSpeed}. By knowing $\mathcal{M}$, the acceleration caused by gas can be written as \cite{tanaka2009assembly, choksi2017recoiling}:
			\begin{equation}\label{eq: df_c}
				a^\text{c}_\text{DF}(\vec{x}, \dot{\vec{x}}) = -\dfrac{4\pi G^2}{\dot{x}^2}M_\bullet\rho_\text{gas}(\vec{x})f(\mathcal{M})
			\end{equation}
			
			with
			\begin{equation}
				f(\mathcal{M}) = \left\{
				\begin{matrix}
				0.5\ln\Lambda \left[\erf{\dfrac{\mathcal{M}}{\sqrt{2}}} - \sqrt{\dfrac{2}{\pi}}\mathcal{M}e^{-\mathcal{M}^2/2}\right]& \text{if $\mathcal{M} \leq 0.8$}\\
				1.5\ln\Lambda \left[\erf{\dfrac{\mathcal{M}}{\sqrt{2}}} - \sqrt{\dfrac{2}{\pi}}\mathcal{M}e^{-\mathcal{M}^2/2}\right] & \text{if $0.8 < \mathcal{M} \leq \mathcal{M}_{eq}$}\\
				0.5\ln\left(1 - \mathcal{M}^{-2}\right) + \ln\Lambda & \text{if $\mathcal{M} > \mathcal{M}_{eq}$}
				\end{matrix}
				\right.
			\end{equation}
			
			$M_{eq}$ is the mach number that fulfills the following equation:
			\begin{equation}\label{eq: machEq}
				\ln\Lambda\left[1.5\left(\erf{\dfrac{\mathcal{M}}{\sqrt{2}}} - \sqrt{\dfrac{2}{\pi}}\mathcal{M}e^{-\mathcal{M}^2/2}\right) - 1\right] - 0.5\ln\left(1 - \mathcal{M}^{-2}\right) = 0
			\end{equation}
			
			Numerically solving \autoref{eq: machEq}, yields $M_{eq} \approx 1.731$ for a value of the Coulomb logarithm $\ln\Lambda = 2.3$. The full acceleration due to dynamical friction is given by the sum of the noncollisional drag on \autoref{eq: df_cl} and \autoref{eq: df_c}:
			\begin{equation}
				a_\text{DF}(\vec{x}, \dot{\vec{x}}) = a_\text{DF}^\text{cl}(\vec{x}, \dot{\vec{x}}) + a_\text{DF}^\text{c}(\vec{x}, \dot{\vec{x}})
			\end{equation}
		
		\subsection{Accretion onto the black hole}
			As the black hole accretes matter from the surroundings, an acceleration appears, due to the second law of Newton:
			\begin{equation}
				\vec{F} = \dfrac{d\vec{P}}{dt} = \dot{\vec{x}}\dot{M}_\bullet + M_\bullet\ddot{\vec{x}}
			\end{equation}
			
			By considering conservation of momentum:
			\begin{equation}
				\ddot{\vec{x}} = - \dot{\vec{x}}\dfrac{\dot{M}_\bullet}{M_\bullet}
			\end{equation}
			
			Two schemes describe the speed at which the black hole gains mass, on the first one the black hole undergoes Bondi-Hoyle-Littleton accretion \cite{tanaka2009assembly, choksi2017recoiling}:
			\begin{equation}
				\dot{M}_\bullet^\text{BHL}(\vec{x}, \dot{\vec{x}}) = \dfrac{4\pi G^2 \rho_G(\vec{x})M^2_\bullet}{\left(c_s^2 + \dot{x}^2\right)^{3/2}} \qquad \text{with } \rho_B(\vec{x}) = \rho_\text{stars}(\vec{x}) + \rho_\text{gas}(\vec{x})
			\end{equation}
			
			There is a limit of accretion for the black hole given by the Eddington luminosity:
			\begin{equation}
				\dot{M}_\bullet^\text{Edd} = \dfrac{(1 - \epsilon)M_\bullet}{\epsilon t_\text{Edd}} \qquad \epsilon = 0.1, \quad t_\text{Edd} = 0.44 \text{ Gyr}
			\end{equation}
			
			Final accretion rate is given by:
			\begin{equation}
				\dot{M}_\bullet(\vec{x}, \dot{\vec{x}}) = \left\{
				\begin{array}{lc}
				\dot{M}_\bullet^\text{BHL}(\vec{x}, \dot{\vec{x}}) & \text{if $\dot{M}_\bullet^\text{BHL} < \dot{M}_\bullet^\text{Edd}$} \\
				\dot{M}_\bullet^\text{Edd} & \text{else}
				\end{array}
				\right.
			\end{equation}
	
	\subsection{Initial conditions and numerical integration}
		For all simulations the virial radius remains constant through the simulation. The virial radius is fixed at the start of every simulation depending on the redshift at which the kick occurs, the chosen densities profiles and the mass of the host galaxy. Sound speed also remains constant for a simulation, as it depends on $R_\text{vir}$ and the mass of the host. Cosmological acceleration is ignored at all times as in \citeauthor{tanaka2009assembly}, as it has been found that it only marginally affects black hole orbits \cite{choksi2017recoiling}. The initial position of the black hole is always $\vec{x} = (0, 0, 0)$ kpc.
		
		Numerical integration is carried out using a leapfrog scheme on REBOUND with the C programming language \cite{larson2017modeling}, with time steps of a thousand years, the simulations are stopped when the system destabilizes and starts gaining energy, due to singularities at $x \rightarrow 0$ and $\dot{x} \rightarrow 0$, or if they simply last more than the age of the universe. 
		
	\section{Definitions}
		\subsection{Escape velocity}
			Minimum initial velocity required for the maximum distance of a single orbit of the black hole to stay outside $0.1R_\text{vir}$ after $z = 0$, $z = 6$ or 10 \% of the age of the universe at the moment of the kick \cite{tanaka2009assembly, choksi2017recoiling}.
		
		\subsection{Time of return}
			Time required by the black hole to orbit with maximum distances of less than  $0.01R_\text{vir}$.
% !TeX spellcheck = en_US
% Chapter 1

%\chapter{Chapter Title Here} % Main chapter title
%
%\label{Chapter1} % For referencing the chapter elsewhere, use \ref{Chapter1} 

%----------------------------------------------------------------------------------------

% Define some commands to keep the formatting separated from the content
%\newcommand{\option}[1]{\texttt{\itshape#1}}

%----------------------------------------------------------------------------------------
\chapter{Results}
	\section{Spherical study}
	For a single simulation, the following data is saved: iteration time, current position, speed, and the black hole mass. With these information, accelerations and densities can be later reconstructed as on \autoref{fig: overallOutput}.
	\begin{figure}[h]
		\centering
		\includegraphics[width = 0.52\textwidth]{"../Files/Week 6/properties_s02v70"}
		\caption{Upper two plots show the output of a single simulation, while the lower one shows most the local properties per data point.}
		\label{fig: overallOutput}
	\end{figure}
	
	\subsection{Effect of the baryonic fraction}
	\begin{figure}[h]
		\centering
		\includegraphics[width = 0.7\textwidth]{"../Files/Week 5/baryonic_fraction_comparison"}
		\caption{Effect of the baryonic fraction in the orbit of the black hole.}
		\label{fig: baryonicfraction}
	\end{figure}
		
	\subsection{Effect of the power law exponent}
	\begin{figure}[h]
		\centering
		\begin{subfigure}[t]{0.49\textwidth}
			\includegraphics[width = \textwidth]{"../Files/Week 6/power_law"}
			\caption{Effect of the power law exponent on the orbits of the black hole.}
			\label{fig: powerLawOrbits}
		\end{subfigure}
		~ 
		\begin{subfigure}[t]{0.49\textwidth}
			\includegraphics[width=\textwidth]{"../Files/Week 6/power_law_density"}
			\caption{Density and mass of the host galaxy as a function of the distance from the center, for different exponents.}
			\label{fig: powerLawDensities}
		\end{subfigure}
		\caption{Properties of the power law exponent.}
		\label{fig: powerLaw}
	\end{figure}
	
	\subsection{Effect of the stellar fraction}
	\begin{figure}[h]
		\centering
		\includegraphics[width = 0.7\textwidth]{"../Files/Week 7/Symmetric/returntimes_stellar_speed"}
		\caption{Return times of the black hole for different initial speeds and stellar fractions.}
		\label{fig: stellarfraction}
	\end{figure}

	\begin{figure}[h]
		\centering
		\includegraphics[width = 0.7\textwidth]{"../Files/Week 7/Symmetric/returntimes_mass"}
		\caption{Return masses of the black hole for different initial speeds and stellar fractions as a function of their return time.}
		\label{fig: returnMass}
	\end{figure}

	\begin{figure}[h]
		\centering
		\includegraphics[width = 0.9\linewidth]{"../Files/Week 9/PhaseSpace_escape"}
		\caption{Phase space generated with different stellar fractions, for an initial velocity $\vec{v} = 90\hat{i}$ kpc/Gyr.}
		\label{fig: escapePhaseSpace}
	\end{figure}
	\begin{figure}[h]
		\centering
		\includegraphics[width = 0.9\linewidth]{"../Files/Week 9/PhaseSpace_in"}
		\caption{Phase space generated with different stellar fractions, for an initial velocity $\vec{v} = 60\hat{i}$ kpc/Gyr.}
		\label{fig: escapeInner}
	\end{figure}

	\section{Triaxial study}
	
	\begin{figure}[h]
		\centering
		\begin{subfigure}[b]{0.49\textwidth}
			\includegraphics[width = \textwidth]{"../Files/Week 7/orthogonal_triaxial"}
			\caption{Orbits for three orthogonal launches}
			\label{fig: orthogonalLaunches}
		\end{subfigure}
		~ 
		\begin{subfigure}[b]{0.49\textwidth}
			\includegraphics[width=\textwidth]{"../Files/Week 7/ellipsoid"}
			\caption{Elliptical geometry}
		\end{subfigure}
		\caption{Orthogonal launches for a triaxial profile with semi-axis ($a_1$:$a_2$:$a_3$) = (1:0.99:0.95)}
		\label{fig: mainOrthogonalLaunches}
	\end{figure}
% Chapter Template

\chapter{Conclusions} % Main chapter title
\label{Conclusions} % Change X to a consecutive number; for referencing this chapter elsewhere, use \ref{ChapterX} 
%% !TeX spellcheck = en_US
% Chapter 1

%\chapter{Chapter Title Here} % Main chapter title
%
%\label{Chapter1} % For referencing the chapter elsewhere, use \ref{Chapter1} 

%----------------------------------------------------------------------------------------

% Define some commands to keep the formatting separated from the content
%\newcommand{\option}[1]{\texttt{\itshape#1}}

%----------------------------------------------------------------------------------------
\chapter{Triaxial study}
	\section{Setup}	
	\section{Results}
	\begin{figure}[h]
		\centering
		\begin{subfigure}[b]{0.49\textwidth}
			\includegraphics[width = \textwidth]{"../Files/Week 7/symmetric"}
			\caption{.}
			\label{fig: symmetricDensity3d}
		\end{subfigure}
		~ 
		\begin{subfigure}[b]{0.49\textwidth}
			\includegraphics[width=\textwidth]{"../Files/Week 7/triaxial"}
			\caption{.}
			\label{fig: triaxialDensity3d}
		\end{subfigure}
		\caption{.}
		\label{fig: symmetricTriaxial}
	\end{figure}

	\begin{equation}
		r_E = \sqrt{\left(\dfrac{x}{a}\right)^2 + \left(\dfrac{y}{b}\right)^2 + \left(\dfrac{z}{c}\right)^2}
	\end{equation}
	
	\begin{equation}
		\tilde{r} = \dfrac{(r_a + r)}{(r_a + r_E)}r
	\end{equation}
	
	$\Phi(x, y, z) = \Phi(\tilde{r})$ \cite{vogelsberger2008fine}
%\include{Chapters/Chapter4} 
%\include{Chapters/Chapter5} 

%----------------------------------------------------------------------------------------
%	THESIS CONTENT - APPENDICES
%----------------------------------------------------------------------------------------

\appendix % Cue to tell LaTeX that the following "chapters" are Appendices

% Include the appendices of the thesis as separate files from the Appendices folder
% Uncomment the lines as you write the Appendices

% !TeX spellcheck = en_US
% Chapter Template

\chapter{Computational setup} % Main chapter title
	\section{Units}
	Computer simulations are sensitive to rounding errors due to the lack of infinite precision when representing decimal numbers. Really small numbers as well as really big ones tend to have bigger errors than those close to the unity, as can be seen on \autoref{fig: IEEE-754}.
	\begin{figure}[h]
		\centering
		\includegraphics[width=0.8\linewidth]{"../Files/Week 3/floating"}
		\caption{Floating point precision for different values, for a 32 bit and 64 bit holders.}
		\label{fig: IEEE-754}
	\end{figure}
	
	Under the International System of Units, distances are measured on meters, times on seconds, and masses on kilograms, nevertheless black holes are too heavy to be measured on kilograms, galaxies sizes too big to be quantified on meters, and time scales too large for a second. Because of that, the following units will be used throughout this document:
	\begin{table}[h]
		\centering
		\caption{Units of measure used on the simulations.}
		\label{tb: units}
		\begin{tabular}{c|c}
			\hline
			\textbf{Physical property} & \textbf{unit} \\
			\hline
			Length & 1 kilo-parsec (kpc) \\
			Mass & $10^5$ solar masses ($10^5$ \sm) \\
			Time & 1 giga-year (Gyr) \\
			\hline
		\end{tabular}
	\end{table}
	
	Along with the change of units, the universal gravitational constant and the Hubble parameter values are required to change.
	
	\subsection{Universal gravitational constant}
	First quantified by Henry Cavendish the gravitational constant has a value of $G_0 = 6.67408\times10^{-11}$ on SI units of m$^3$s$^{-2}$kg$^{-1}$. With the units of length, mass and time on \autoref{tb: units}, the constant of gravity used is given by:
	\begin{equation}
	\begin{array}{ccl}
	G & = & G_0 \left(\dfrac{1 \text{ kpc}^3}{\left(3.0857\times10^{19}\right)^3  \text{ m}^3}\right)\left(\dfrac{\left(3.154\times10^{16}\right)^2 \text{ s}^2}{1 \text{ Gyr}^2}\right)\left(\dfrac{1.98847\times10^{35} \text{ kg}}{10^5 M_\theta}\right) \\
	& = & 0.4493 \quad \dfrac{\text{kpc$^3$}}{\text{Gy$r^210^5$\sm}}	
	\end{array}
	\end{equation}
	
	
	\subsection{Hubble parameter}
	The Hubble constant is frequently used as $H_0 = 67.66 \pm 0.42$ kms$^{-1}$Mpc$^{-1}$ \cite{aghanim2018planck}, stating the speed of an astronomical body on kms$^{-1}$ at a distance of 1 Mpc. Nevertheless, the hubble constant has units of 1/time, thus, taking into account the units on \autoref{tb: units} one gets:
	\begin{equation}
	\begin{array}{ccl}
	H & = & H_0 \left(\dfrac{1 \text{ kpc}}{3.0857\times10^{16} \text{ km}}\right)\left(\dfrac{3.154\times10^{16} \text{ s}}{1 \text{ Gyr}}\right)\left(\dfrac{1 \text{ Mpc}}{1000 \text{ kpc}}\right) \\
	& \approx & 1.023 H_0 \times10^{-3} \text{ Gyr$^{-1}$} \\ 
	& = & 6.916\times10^{-2}\text{ Gyr$^{-1}$}
	\end{array}
	\end{equation}
	
	Although the Hubble parameter is often called Hubble constant, its value changes with time as can be seen on \autoref{fig: hubbleTime}. %In particular, at $z = 20$, the moment at which the kick occurs $H$ has a value of 3.699 Gyr$^{-1}$.
	\begin{figure}[h]
		\centering
		\includegraphics[width=0.8\linewidth]{"../Files/Week 5/hubble_time"}
		\caption{Dependency of the Hubble parameter with redshift.}
		\label{fig: hubbleTime}
	\end{figure}
	
	\section{Critical density and Virial Radius}\label{sec: cd_vr}
	Mass distributions used for the simulation of the host galaxy, are divergent for distances up to infinity. Because of this, the cumulative mass of all bodies within a given distance is called the virial mass and its value is taken as the mass of the whole system. The distance taken to calculate the virial mass is called virial radius ($R_\text{vir}$), and it is defined as the distance at which the average density of the galaxy is 200 times the critical density of the universe ($\rho_\text{crit}$).
	\begin{equation}\label{eq: critical_density}
	\rho_\text{crit} = \dfrac{3H(t)^2}{8\pi G}
	\end{equation}
	
	\begin{equation}\label{eq: R_vir_def}
	\begin{array}{c}
	\dfrac{M(R_\text{vir})}{V(R_\text{vir})} = \bar{\rho}(R_\text{vir}) =  200 \rho_\text{crit} = 75\dfrac{H(t)^2}{\pi G}\\
	\text{where $M(R_\text{vir})$ is the cumulative mass, and $V(R_\text{vir})$: the volume}
	\end{array}			
	\end{equation}
	
	The relation on \autoref{eq: critical_density} is found by considering the case where the geometry of the universe is flat, as a consequence it is said that the critical density is the minimum density required to stop the expansion of the universe \cite{binney2011galactic}.
	
\chapter{Lyapunov exponents}
	In chaotic behavior, infinitesimally close initial conditions lead to evolutions that diverge exponentially fast. The Maximum Lyapunov Exponent $\mathcal{L}$, is an indicative of the rate of such divergence.
	\begin{figure}[h]
		\centering
		\includegraphics[width = 0.7\linewidth]{"../Files/Week 9/lyapunov_explain"}
		\caption{Representation of three arbitrary close orbits, and their evolution in time.}
		\label{fig: lyapunov_explain}
	\end{figure}
	
	Consider the upper two orbits ($\mathcal{O}^\text{ref}$, $\mathcal{O}^\text{var}$) in \autoref{fig: lyapunov_explain}, with initial conditions $\vec{x}^\text{ ref}(0)$, $\vec{p}^\text{ ref}(0)$ and $\vec{x}^\text{ var}(0)$, $\vec{p}^\text{ var}(0)$. Denoting the distance in each of the components of the phase space as:
	\begin{equation}\label{eq: deltax}
		\delta\vec{x}(t) = \vec{x}^\text{ ref}(t) - \vec{x}^\text{ var}(t) = \left(x^\text{ref}(t) - x^\text{var}(t), y^\text{ref}(t) - y^\text{var}(t), z^\text{ref}(t) - z^\text{var}(t)\right)
	\end{equation}
	\begin{equation}\label{eq: deltap}
		\delta\vec{p}(t) = \vec{p}^\text{ ref}(t) - \vec{p}^\text{ var}(t) = \left(p_x^\text{ref}(t) - p_x^\text{var}(t), p_y^\text{ref}(t) - p_y^\text{var}(t), p_z^\text{ref}(t) - p_z^\text{var}(t)\right)
	\end{equation}
	
	the Maximum Lyapunov Exponent can be written as:
	\begin{equation}
		\mathcal{L} = \lim_{t\rightarrow\infty}\dfrac{1}{t}\ln\dfrac{\left|\delta\vec{x}(t), \delta\vec{p}(t)\right|}{\left|\delta\vec{x}(0), \delta\vec{p}(0)\right|}
	\end{equation}
	
	where $|\delta\vec{x}(t), \delta\vec{p}(t)|$ is the Euclidean norm of the 6 dimensional phase space. The numerical calculation of $\mathcal{L}$ requires special care, as a computation up to infinity must be done. In 1980 a technique by Benetti solved this problem, as The Maximum Lyapunov Exponent can be calculated as follows:
	\begin{enumerate}
		\item Define an arbitrary initial distance in the phase space $\delta\vec{x}(0)$, $\delta\vec{p}(0) \equiv 0$.
		\item Simulate both $\mathcal{O}^\text{ref}$ and $\mathcal{O}^\text{var}$ until a predefined time $T$.
		\item Calculate the distance in phase space at time $T$ between the reference orbit and the variational one (equations \ref{eq: deltax} and \ref{eq: deltap}).
		\item Calculate the coefficient $s_i$.
		\begin{equation}
			s_i = \dfrac{\left|\delta\vec{x}_i(T), \delta\vec{p}_i(T)\right|}{\left|\delta_i\vec{x}(0)\right|}
		\end{equation}
	\end{enumerate} 
	
\chapter{Galaxies}\label{ch: galaxies}

\begin{figure}[h]
    \centering
    \begin{subfigure}[t]{0.4\textwidth}
        \includegraphics[width = \textwidth]{"../Files/Week 13/images/21_time"}
        \caption{Return times}
    \end{subfigure}
    ~ 
    \begin{subfigure}[t]{0.4\textwidth}
        \includegraphics[width=\textwidth]{"../Files/Week 13/images/21_mass"}
        \caption{Return masses}
    \end{subfigure}
    \begin{subfigure}[t]{0.4\textwidth}
        \includegraphics[width=\textwidth]{"../Files/Week 13/images/21_lyapunov"}
        \caption{Lyapunov exponent}
    \end{subfigure}
    \begin{subfigure}[t]{0.4\textwidth}
        \includegraphics[width=\textwidth]{"../Files/Week 13/images/21_ellipsoid"}
        \caption{Geometry}
    \end{subfigure}
    \caption{Distribution of the different properties for the galaxy with $a_1 = 1$, $a_2 = 1.0e+00$, $a_3 = 4.6\times10^{-1}$.}
\end{figure}


\begin{figure}[h]
    \centering
    \begin{subfigure}[t]{0.4\textwidth}
        \includegraphics[width = \textwidth]{"../Files/Week 13/images/26_time"}
        \caption{Return times}
    \end{subfigure}
    ~ 
    \begin{subfigure}[t]{0.4\textwidth}
        \includegraphics[width=\textwidth]{"../Files/Week 13/images/26_mass"}
        \caption{Return masses}
    \end{subfigure}
    \begin{subfigure}[t]{0.4\textwidth}
        \includegraphics[width=\textwidth]{"../Files/Week 13/images/26_lyapunov"}
        \caption{Lyapunov exponent}
    \end{subfigure}
    \begin{subfigure}[t]{0.4\textwidth}
        \includegraphics[width=\textwidth]{"../Files/Week 13/images/26_ellipsoid"}
        \caption{Geometry}
    \end{subfigure}
    \caption{Distribution of the different properties for the galaxy with $a_1 = 1$, $a_2 = 9.3\times10^{-1}$, $a_3 = 2.6\times10^{-1}$.}
\end{figure}


\begin{figure}[h]
    \centering
    \begin{subfigure}[t]{0.4\textwidth}
        \includegraphics[width = \textwidth]{"../Files/Week 13/images/10_time"}
        \caption{Return times}
    \end{subfigure}
    ~ 
    \begin{subfigure}[t]{0.4\textwidth}
        \includegraphics[width=\textwidth]{"../Files/Week 13/images/10_mass"}
        \caption{Return masses}
    \end{subfigure}
    \begin{subfigure}[t]{0.4\textwidth}
        \includegraphics[width=\textwidth]{"../Files/Week 13/images/10_lyapunov"}
        \caption{Lyapunov exponent}
    \end{subfigure}
    \begin{subfigure}[t]{0.4\textwidth}
        \includegraphics[width=\textwidth]{"../Files/Week 13/images/10_ellipsoid"}
        \caption{Geometry}
    \end{subfigure}
    \caption{Distribution of the different properties for the galaxy with $a_1 = 1$, $a_2 = 9.6\times10^{-1}$, $a_3 = 7.0\times10^{-1}$.}
\end{figure}


\begin{figure}[h]
    \centering
    \begin{subfigure}[t]{0.4\textwidth}
        \includegraphics[width = \textwidth]{"../Files/Week 13/images/17_time"}
        \caption{Return times}
    \end{subfigure}
    ~ 
    \begin{subfigure}[t]{0.4\textwidth}
        \includegraphics[width=\textwidth]{"../Files/Week 13/images/17_mass"}
        \caption{Return masses}
    \end{subfigure}
    \begin{subfigure}[t]{0.4\textwidth}
        \includegraphics[width=\textwidth]{"../Files/Week 13/images/17_lyapunov"}
        \caption{Lyapunov exponent}
    \end{subfigure}
    \begin{subfigure}[t]{0.4\textwidth}
        \includegraphics[width=\textwidth]{"../Files/Week 13/images/17_ellipsoid"}
        \caption{Geometry}
    \end{subfigure}
    \caption{Distribution of the different properties for the galaxy with $a_1 = 1$, $a_2 = 8.7\times10^{-1}$, $a_3 = 1.3\times10^{-1}$.}
\end{figure}


\begin{figure}[h]
    \centering
    \begin{subfigure}[t]{0.4\textwidth}
        \includegraphics[width = \textwidth]{"../Files/Week 13/images/3_time"}
        \caption{Return times}
    \end{subfigure}
    ~ 
    \begin{subfigure}[t]{0.4\textwidth}
        \includegraphics[width=\textwidth]{"../Files/Week 13/images/3_mass"}
        \caption{Return masses}
    \end{subfigure}
    \begin{subfigure}[t]{0.4\textwidth}
        \includegraphics[width=\textwidth]{"../Files/Week 13/images/3_lyapunov"}
        \caption{Lyapunov exponent}
    \end{subfigure}
    \begin{subfigure}[t]{0.4\textwidth}
        \includegraphics[width=\textwidth]{"../Files/Week 13/images/3_ellipsoid"}
        \caption{Geometry}
    \end{subfigure}
    \caption{Distribution of the different properties for the galaxy with $a_1 = 1$, $a_2 = 6.9\times10^{-1}$, $a_3 = 1.2\times10^{-2}$.}
\end{figure}


\begin{figure}[h]
    \centering
    \begin{subfigure}[t]{0.4\textwidth}
        \includegraphics[width = \textwidth]{"../Files/Week 13/images/28_time"}
        \caption{Return times}
    \end{subfigure}
    ~ 
    \begin{subfigure}[t]{0.4\textwidth}
        \includegraphics[width=\textwidth]{"../Files/Week 13/images/28_mass"}
        \caption{Return masses}
    \end{subfigure}
    \begin{subfigure}[t]{0.4\textwidth}
        \includegraphics[width=\textwidth]{"../Files/Week 13/images/28_lyapunov"}
        \caption{Lyapunov exponent}
    \end{subfigure}
    \begin{subfigure}[t]{0.4\textwidth}
        \includegraphics[width=\textwidth]{"../Files/Week 13/images/28_ellipsoid"}
        \caption{Geometry}
    \end{subfigure}
    \caption{Distribution of the different properties for the galaxy with $a_1 = 1$, $a_2 = 6.8\times10^{-1}$, $a_3 = 2.2\times10^{-1}$.}
\end{figure}


\begin{figure}[h]
    \centering
    \begin{subfigure}[t]{0.4\textwidth}
        \includegraphics[width = \textwidth]{"../Files/Week 13/images/20_time"}
        \caption{Return times}
    \end{subfigure}
    ~ 
    \begin{subfigure}[t]{0.4\textwidth}
        \includegraphics[width=\textwidth]{"../Files/Week 13/images/20_mass"}
        \caption{Return masses}
    \end{subfigure}
    \begin{subfigure}[t]{0.4\textwidth}
        \includegraphics[width=\textwidth]{"../Files/Week 13/images/20_lyapunov"}
        \caption{Lyapunov exponent}
    \end{subfigure}
    \begin{subfigure}[t]{0.4\textwidth}
        \includegraphics[width=\textwidth]{"../Files/Week 13/images/20_ellipsoid"}
        \caption{Geometry}
    \end{subfigure}
    \caption{Distribution of the different properties for the galaxy with $a_1 = 1$, $a_2 = 6.3\times10^{-1}$, $a_3 = 2.8\times10^{-1}$.}
\end{figure}


\begin{figure}[h]
    \centering
    \begin{subfigure}[t]{0.4\textwidth}
        \includegraphics[width = \textwidth]{"../Files/Week 13/images/19_time"}
        \caption{Return times}
    \end{subfigure}
    ~ 
    \begin{subfigure}[t]{0.4\textwidth}
        \includegraphics[width=\textwidth]{"../Files/Week 13/images/19_mass"}
        \caption{Return masses}
    \end{subfigure}
    \begin{subfigure}[t]{0.4\textwidth}
        \includegraphics[width=\textwidth]{"../Files/Week 13/images/19_lyapunov"}
        \caption{Lyapunov exponent}
    \end{subfigure}
    \begin{subfigure}[t]{0.4\textwidth}
        \includegraphics[width=\textwidth]{"../Files/Week 13/images/19_ellipsoid"}
        \caption{Geometry}
    \end{subfigure}
    \caption{Distribution of the different properties for the galaxy with $a_1 = 1$, $a_2 = 5.7\times10^{-1}$, $a_3 = 3.3\times10^{-1}$.}
\end{figure}


\begin{figure}[h]
    \centering
    \begin{subfigure}[t]{0.4\textwidth}
        \includegraphics[width = \textwidth]{"../Files/Week 13/images/1_time"}
        \caption{Return times}
    \end{subfigure}
    ~ 
    \begin{subfigure}[t]{0.4\textwidth}
        \includegraphics[width=\textwidth]{"../Files/Week 13/images/1_mass"}
        \caption{Return masses}
    \end{subfigure}
    \begin{subfigure}[t]{0.4\textwidth}
        \includegraphics[width=\textwidth]{"../Files/Week 13/images/1_lyapunov"}
        \caption{Lyapunov exponent}
    \end{subfigure}
    \begin{subfigure}[t]{0.4\textwidth}
        \includegraphics[width=\textwidth]{"../Files/Week 13/images/1_ellipsoid"}
        \caption{Geometry}
    \end{subfigure}
    \caption{Distribution of the different properties for the galaxy with $a_1 = 1$, $a_2 = 4.8\times10^{-1}$, $a_3 = 3.1\times10^{-2}$.}
\end{figure}


\begin{figure}[h]
    \centering
    \begin{subfigure}[t]{0.4\textwidth}
        \includegraphics[width = \textwidth]{"../Files/Week 13/images/5_time"}
        \caption{Return times}
    \end{subfigure}
    ~ 
    \begin{subfigure}[t]{0.4\textwidth}
        \includegraphics[width=\textwidth]{"../Files/Week 13/images/5_mass"}
        \caption{Return masses}
    \end{subfigure}
    \begin{subfigure}[t]{0.4\textwidth}
        \includegraphics[width=\textwidth]{"../Files/Week 13/images/5_lyapunov"}
        \caption{Lyapunov exponent}
    \end{subfigure}
    \begin{subfigure}[t]{0.4\textwidth}
        \includegraphics[width=\textwidth]{"../Files/Week 13/images/5_ellipsoid"}
        \caption{Geometry}
    \end{subfigure}
    \caption{Distribution of the different properties for the galaxy with $a_1 = 1$, $a_2 = 6.4\times10^{-1}$, $a_3 = 4.8\times10^{-1}$.}
\end{figure}


\begin{figure}[h]
    \centering
    \begin{subfigure}[t]{0.4\textwidth}
        \includegraphics[width = \textwidth]{"../Files/Week 13/images/29_time"}
        \caption{Return times}
    \end{subfigure}
    ~ 
    \begin{subfigure}[t]{0.4\textwidth}
        \includegraphics[width=\textwidth]{"../Files/Week 13/images/29_mass"}
        \caption{Return masses}
    \end{subfigure}
    \begin{subfigure}[t]{0.4\textwidth}
        \includegraphics[width=\textwidth]{"../Files/Week 13/images/29_lyapunov"}
        \caption{Lyapunov exponent}
    \end{subfigure}
    \begin{subfigure}[t]{0.4\textwidth}
        \includegraphics[width=\textwidth]{"../Files/Week 13/images/29_ellipsoid"}
        \caption{Geometry}
    \end{subfigure}
    \caption{Distribution of the different properties for the galaxy with $a_1 = 1$, $a_2 = 5.0\times10^{-1}$, $a_3 = 1.7\times10^{-1}$.}
\end{figure}


\begin{figure}[h]
    \centering
    \begin{subfigure}[t]{0.4\textwidth}
        \includegraphics[width = \textwidth]{"../Files/Week 13/images/7_time"}
        \caption{Return times}
    \end{subfigure}
    ~ 
    \begin{subfigure}[t]{0.4\textwidth}
        \includegraphics[width=\textwidth]{"../Files/Week 13/images/7_mass"}
        \caption{Return masses}
    \end{subfigure}
    \begin{subfigure}[t]{0.4\textwidth}
        \includegraphics[width=\textwidth]{"../Files/Week 13/images/7_lyapunov"}
        \caption{Lyapunov exponent}
    \end{subfigure}
    \begin{subfigure}[t]{0.4\textwidth}
        \includegraphics[width=\textwidth]{"../Files/Week 13/images/7_ellipsoid"}
        \caption{Geometry}
    \end{subfigure}
    \caption{Distribution of the different properties for the galaxy with $a_1 = 1$, $a_2 = 5.8\times10^{-1}$, $a_3 = 4.1\times10^{-1}$.}
\end{figure}


\begin{figure}[h]
    \centering
    \begin{subfigure}[t]{0.4\textwidth}
        \includegraphics[width = \textwidth]{"../Files/Week 13/images/13_time"}
        \caption{Return times}
    \end{subfigure}
    ~ 
    \begin{subfigure}[t]{0.4\textwidth}
        \includegraphics[width=\textwidth]{"../Files/Week 13/images/13_mass"}
        \caption{Return masses}
    \end{subfigure}
    \begin{subfigure}[t]{0.4\textwidth}
        \includegraphics[width=\textwidth]{"../Files/Week 13/images/13_lyapunov"}
        \caption{Lyapunov exponent}
    \end{subfigure}
    \begin{subfigure}[t]{0.4\textwidth}
        \includegraphics[width=\textwidth]{"../Files/Week 13/images/13_ellipsoid"}
        \caption{Geometry}
    \end{subfigure}
    \caption{Distribution of the different properties for the galaxy with $a_1 = 1$, $a_2 = 7.5\times10^{-1}$, $a_3 = 6.8\times10^{-1}$.}
\end{figure}


\begin{figure}[h]
    \centering
    \begin{subfigure}[t]{0.4\textwidth}
        \includegraphics[width = \textwidth]{"../Files/Week 13/images/9_time"}
        \caption{Return times}
    \end{subfigure}
    ~ 
    \begin{subfigure}[t]{0.4\textwidth}
        \includegraphics[width=\textwidth]{"../Files/Week 13/images/9_mass"}
        \caption{Return masses}
    \end{subfigure}
    \begin{subfigure}[t]{0.4\textwidth}
        \includegraphics[width=\textwidth]{"../Files/Week 13/images/9_lyapunov"}
        \caption{Lyapunov exponent}
    \end{subfigure}
    \begin{subfigure}[t]{0.4\textwidth}
        \includegraphics[width=\textwidth]{"../Files/Week 13/images/9_ellipsoid"}
        \caption{Geometry}
    \end{subfigure}
    \caption{Distribution of the different properties for the galaxy with $a_1 = 1$, $a_2 = 4.2\times10^{-1}$, $a_3 = 7.4\times10^{-2}$.}
\end{figure}


\begin{figure}[h]
    \centering
    \begin{subfigure}[t]{0.4\textwidth}
        \includegraphics[width = \textwidth]{"../Files/Week 13/images/2_time"}
        \caption{Return times}
    \end{subfigure}
    ~ 
    \begin{subfigure}[t]{0.4\textwidth}
        \includegraphics[width=\textwidth]{"../Files/Week 13/images/2_mass"}
        \caption{Return masses}
    \end{subfigure}
    \begin{subfigure}[t]{0.4\textwidth}
        \includegraphics[width=\textwidth]{"../Files/Week 13/images/2_lyapunov"}
        \caption{Lyapunov exponent}
    \end{subfigure}
    \begin{subfigure}[t]{0.4\textwidth}
        \includegraphics[width=\textwidth]{"../Files/Week 13/images/2_ellipsoid"}
        \caption{Geometry}
    \end{subfigure}
    \caption{Distribution of the different properties for the galaxy with $a_1 = 1$, $a_2 = 6.4\times10^{-1}$, $a_3 = 5.5\times10^{-1}$.}
\end{figure}


\begin{figure}[h]
    \centering
    \begin{subfigure}[t]{0.4\textwidth}
        \includegraphics[width = \textwidth]{"../Files/Week 13/images/27_time"}
        \caption{Return times}
    \end{subfigure}
    ~ 
    \begin{subfigure}[t]{0.4\textwidth}
        \includegraphics[width=\textwidth]{"../Files/Week 13/images/27_mass"}
        \caption{Return masses}
    \end{subfigure}
    \begin{subfigure}[t]{0.4\textwidth}
        \includegraphics[width=\textwidth]{"../Files/Week 13/images/27_lyapunov"}
        \caption{Lyapunov exponent}
    \end{subfigure}
    \begin{subfigure}[t]{0.4\textwidth}
        \includegraphics[width=\textwidth]{"../Files/Week 13/images/27_ellipsoid"}
        \caption{Geometry}
    \end{subfigure}
    \caption{Distribution of the different properties for the galaxy with $a_1 = 1$, $a_2 = 4.8\times10^{-1}$, $a_3 = 2.9\times10^{-1}$.}
\end{figure}


\begin{figure}[h]
    \centering
    \begin{subfigure}[t]{0.4\textwidth}
        \includegraphics[width = \textwidth]{"../Files/Week 13/images/25_time"}
        \caption{Return times}
    \end{subfigure}
    ~ 
    \begin{subfigure}[t]{0.4\textwidth}
        \includegraphics[width=\textwidth]{"../Files/Week 13/images/25_mass"}
        \caption{Return masses}
    \end{subfigure}
    \begin{subfigure}[t]{0.4\textwidth}
        \includegraphics[width=\textwidth]{"../Files/Week 13/images/25_lyapunov"}
        \caption{Lyapunov exponent}
    \end{subfigure}
    \begin{subfigure}[t]{0.4\textwidth}
        \includegraphics[width=\textwidth]{"../Files/Week 13/images/25_ellipsoid"}
        \caption{Geometry}
    \end{subfigure}
    \caption{Distribution of the different properties for the galaxy with $a_1 = 1$, $a_2 = 3.9\times10^{-1}$, $a_3 = 5.2\times10^{-2}$.}
\end{figure}


\begin{figure}[h]
    \centering
    \begin{subfigure}[t]{0.4\textwidth}
        \includegraphics[width = \textwidth]{"../Files/Week 13/images/12_time"}
        \caption{Return times}
    \end{subfigure}
    ~ 
    \begin{subfigure}[t]{0.4\textwidth}
        \includegraphics[width=\textwidth]{"../Files/Week 13/images/12_mass"}
        \caption{Return masses}
    \end{subfigure}
    \begin{subfigure}[t]{0.4\textwidth}
        \includegraphics[width=\textwidth]{"../Files/Week 13/images/12_lyapunov"}
        \caption{Lyapunov exponent}
    \end{subfigure}
    \begin{subfigure}[t]{0.4\textwidth}
        \includegraphics[width=\textwidth]{"../Files/Week 13/images/12_ellipsoid"}
        \caption{Geometry}
    \end{subfigure}
    \caption{Distribution of the different properties for the galaxy with $a_1 = 1$, $a_2 = 7.5\times10^{-1}$, $a_3 = 7.1\times10^{-1}$.}
\end{figure}


\begin{figure}[h]
    \centering
    \begin{subfigure}[t]{0.4\textwidth}
        \includegraphics[width = \textwidth]{"../Files/Week 13/images/16_time"}
        \caption{Return times}
    \end{subfigure}
    ~ 
    \begin{subfigure}[t]{0.4\textwidth}
        \includegraphics[width=\textwidth]{"../Files/Week 13/images/16_mass"}
        \caption{Return masses}
    \end{subfigure}
    \begin{subfigure}[t]{0.4\textwidth}
        \includegraphics[width=\textwidth]{"../Files/Week 13/images/16_lyapunov"}
        \caption{Lyapunov exponent}
    \end{subfigure}
    \begin{subfigure}[t]{0.4\textwidth}
        \includegraphics[width=\textwidth]{"../Files/Week 13/images/16_ellipsoid"}
        \caption{Geometry}
    \end{subfigure}
    \caption{Distribution of the different properties for the galaxy with $a_1 = 1$, $a_2 = 3.7\times10^{-1}$, $a_3 = 2.8\times10^{-1}$.}
\end{figure}


\begin{figure}[h]
    \centering
    \begin{subfigure}[t]{0.4\textwidth}
        \includegraphics[width = \textwidth]{"../Files/Week 13/images/14_time"}
        \caption{Return times}
    \end{subfigure}
    ~ 
    \begin{subfigure}[t]{0.4\textwidth}
        \includegraphics[width=\textwidth]{"../Files/Week 13/images/14_mass"}
        \caption{Return masses}
    \end{subfigure}
    \begin{subfigure}[t]{0.4\textwidth}
        \includegraphics[width=\textwidth]{"../Files/Week 13/images/14_lyapunov"}
        \caption{Lyapunov exponent}
    \end{subfigure}
    \begin{subfigure}[t]{0.4\textwidth}
        \includegraphics[width=\textwidth]{"../Files/Week 13/images/14_ellipsoid"}
        \caption{Geometry}
    \end{subfigure}
    \caption{Distribution of the different properties for the galaxy with $a_1 = 1$, $a_2 = 2.8\times10^{-1}$, $a_3 = 1.5\times10^{-1}$.}
\end{figure}


\begin{figure}[h]
    \centering
    \begin{subfigure}[t]{0.4\textwidth}
        \includegraphics[width = \textwidth]{"../Files/Week 13/images/8_time"}
        \caption{Return times}
    \end{subfigure}
    ~ 
    \begin{subfigure}[t]{0.4\textwidth}
        \includegraphics[width=\textwidth]{"../Files/Week 13/images/8_mass"}
        \caption{Return masses}
    \end{subfigure}
    \begin{subfigure}[t]{0.4\textwidth}
        \includegraphics[width=\textwidth]{"../Files/Week 13/images/8_lyapunov"}
        \caption{Lyapunov exponent}
    \end{subfigure}
    \begin{subfigure}[t]{0.4\textwidth}
        \includegraphics[width=\textwidth]{"../Files/Week 13/images/8_ellipsoid"}
        \caption{Geometry}
    \end{subfigure}
    \caption{Distribution of the different properties for the galaxy with $a_1 = 1$, $a_2 = 3.2\times10^{-1}$, $a_3 = 2.4\times10^{-1}$.}
\end{figure}


\begin{figure}[h]
    \centering
    \begin{subfigure}[t]{0.4\textwidth}
        \includegraphics[width = \textwidth]{"../Files/Week 13/images/6_time"}
        \caption{Return times}
    \end{subfigure}
    ~ 
    \begin{subfigure}[t]{0.4\textwidth}
        \includegraphics[width=\textwidth]{"../Files/Week 13/images/6_mass"}
        \caption{Return masses}
    \end{subfigure}
    \begin{subfigure}[t]{0.4\textwidth}
        \includegraphics[width=\textwidth]{"../Files/Week 13/images/6_lyapunov"}
        \caption{Lyapunov exponent}
    \end{subfigure}
    \begin{subfigure}[t]{0.4\textwidth}
        \includegraphics[width=\textwidth]{"../Files/Week 13/images/6_ellipsoid"}
        \caption{Geometry}
    \end{subfigure}
    \caption{Distribution of the different properties for the galaxy with $a_1 = 1$, $a_2 = 2.7\times10^{-1}$, $a_3 = 2.0\times10^{-1}$.}
\end{figure}


\begin{figure}[h]
    \centering
    \begin{subfigure}[t]{0.4\textwidth}
        \includegraphics[width = \textwidth]{"../Files/Week 13/images/0_time"}
        \caption{Return times}
    \end{subfigure}
    ~ 
    \begin{subfigure}[t]{0.4\textwidth}
        \includegraphics[width=\textwidth]{"../Files/Week 13/images/0_mass"}
        \caption{Return masses}
    \end{subfigure}
    \begin{subfigure}[t]{0.4\textwidth}
        \includegraphics[width=\textwidth]{"../Files/Week 13/images/0_lyapunov"}
        \caption{Lyapunov exponent}
    \end{subfigure}
    \begin{subfigure}[t]{0.4\textwidth}
        \includegraphics[width=\textwidth]{"../Files/Week 13/images/0_ellipsoid"}
        \caption{Geometry}
    \end{subfigure}
    \caption{Distribution of the different properties for the galaxy with $a_1 = 1$, $a_2 = 4.8\times10^{-1}$, $a_3 = 4.6\times10^{-1}$.}
\end{figure}


\begin{figure}[h]
    \centering
    \begin{subfigure}[t]{0.4\textwidth}
        \includegraphics[width = \textwidth]{"../Files/Week 13/images/23_time"}
        \caption{Return times}
    \end{subfigure}
    ~ 
    \begin{subfigure}[t]{0.4\textwidth}
        \includegraphics[width=\textwidth]{"../Files/Week 13/images/23_mass"}
        \caption{Return masses}
    \end{subfigure}
    \begin{subfigure}[t]{0.4\textwidth}
        \includegraphics[width=\textwidth]{"../Files/Week 13/images/23_lyapunov"}
        \caption{Lyapunov exponent}
    \end{subfigure}
    \begin{subfigure}[t]{0.4\textwidth}
        \includegraphics[width=\textwidth]{"../Files/Week 13/images/23_ellipsoid"}
        \caption{Geometry}
    \end{subfigure}
    \caption{Distribution of the different properties for the galaxy with $a_1 = 1$, $a_2 = 4.8\times10^{-1}$, $a_3 = 4.6\times10^{-1}$.}
\end{figure}


\begin{figure}[h]
    \centering
    \begin{subfigure}[t]{0.4\textwidth}
        \includegraphics[width = \textwidth]{"../Files/Week 13/images/15_time"}
        \caption{Return times}
    \end{subfigure}
    ~ 
    \begin{subfigure}[t]{0.4\textwidth}
        \includegraphics[width=\textwidth]{"../Files/Week 13/images/15_mass"}
        \caption{Return masses}
    \end{subfigure}
    \begin{subfigure}[t]{0.4\textwidth}
        \includegraphics[width=\textwidth]{"../Files/Week 13/images/15_lyapunov"}
        \caption{Lyapunov exponent}
    \end{subfigure}
    \begin{subfigure}[t]{0.4\textwidth}
        \includegraphics[width=\textwidth]{"../Files/Week 13/images/15_ellipsoid"}
        \caption{Geometry}
    \end{subfigure}
    \caption{Distribution of the different properties for the galaxy with $a_1 = 1$, $a_2 = 3.5\times10^{-1}$, $a_3 = 3.3\times10^{-1}$.}
\end{figure}


\begin{figure}[h]
    \centering
    \begin{subfigure}[t]{0.4\textwidth}
        \includegraphics[width = \textwidth]{"../Files/Week 13/images/18_time"}
        \caption{Return times}
    \end{subfigure}
    ~ 
    \begin{subfigure}[t]{0.4\textwidth}
        \includegraphics[width=\textwidth]{"../Files/Week 13/images/18_mass"}
        \caption{Return masses}
    \end{subfigure}
    \begin{subfigure}[t]{0.4\textwidth}
        \includegraphics[width=\textwidth]{"../Files/Week 13/images/18_lyapunov"}
        \caption{Lyapunov exponent}
    \end{subfigure}
    \begin{subfigure}[t]{0.4\textwidth}
        \includegraphics[width=\textwidth]{"../Files/Week 13/images/18_ellipsoid"}
        \caption{Geometry}
    \end{subfigure}
    \caption{Distribution of the different properties for the galaxy with $a_1 = 1$, $a_2 = 1.2\times10^{-1}$, $a_3 = 3.2\times10^{-3}$.}
\end{figure}


\begin{figure}[h]
    \centering
    \begin{subfigure}[t]{0.4\textwidth}
        \includegraphics[width = \textwidth]{"../Files/Week 13/images/4_time"}
        \caption{Return times}
    \end{subfigure}
    ~ 
    \begin{subfigure}[t]{0.4\textwidth}
        \includegraphics[width=\textwidth]{"../Files/Week 13/images/4_mass"}
        \caption{Return masses}
    \end{subfigure}
    \begin{subfigure}[t]{0.4\textwidth}
        \includegraphics[width=\textwidth]{"../Files/Week 13/images/4_lyapunov"}
        \caption{Lyapunov exponent}
    \end{subfigure}
    \begin{subfigure}[t]{0.4\textwidth}
        \includegraphics[width=\textwidth]{"../Files/Week 13/images/4_ellipsoid"}
        \caption{Geometry}
    \end{subfigure}
    \caption{Distribution of the different properties for the galaxy with $a_1 = 1$, $a_2 = 1.1\times10^{-1}$, $a_3 = 5.3\times10^{-2}$.}
\end{figure}


\begin{figure}[h]
    \centering
    \begin{subfigure}[t]{0.4\textwidth}
        \includegraphics[width = \textwidth]{"../Files/Week 13/images/11_time"}
        \caption{Return times}
    \end{subfigure}
    ~ 
    \begin{subfigure}[t]{0.4\textwidth}
        \includegraphics[width=\textwidth]{"../Files/Week 13/images/11_mass"}
        \caption{Return masses}
    \end{subfigure}
    \begin{subfigure}[t]{0.4\textwidth}
        \includegraphics[width=\textwidth]{"../Files/Week 13/images/11_lyapunov"}
        \caption{Lyapunov exponent}
    \end{subfigure}
    \begin{subfigure}[t]{0.4\textwidth}
        \includegraphics[width=\textwidth]{"../Files/Week 13/images/11_ellipsoid"}
        \caption{Geometry}
    \end{subfigure}
    \caption{Distribution of the different properties for the galaxy with $a_1 = 1$, $a_2 = 2.0\times10^{-1}$, $a_3 = 1.9\times10^{-1}$.}
\end{figure}


\begin{figure}[h]
    \centering
    \begin{subfigure}[t]{0.4\textwidth}
        \includegraphics[width = \textwidth]{"../Files/Week 13/images/22_time"}
        \caption{Return times}
    \end{subfigure}
    ~ 
    \begin{subfigure}[t]{0.4\textwidth}
        \includegraphics[width=\textwidth]{"../Files/Week 13/images/22_mass"}
        \caption{Return masses}
    \end{subfigure}
    \begin{subfigure}[t]{0.4\textwidth}
        \includegraphics[width=\textwidth]{"../Files/Week 13/images/22_lyapunov"}
        \caption{Lyapunov exponent}
    \end{subfigure}
    \begin{subfigure}[t]{0.4\textwidth}
        \includegraphics[width=\textwidth]{"../Files/Week 13/images/22_ellipsoid"}
        \caption{Geometry}
    \end{subfigure}
    \caption{Distribution of the different properties for the galaxy with $a_1 = 1$, $a_2 = 3.7\times10^{-2}$, $a_3 = 2.4\times10^{-2}$.}
\end{figure}


\begin{figure}[h]
    \centering
    \begin{subfigure}[t]{0.4\textwidth}
        \includegraphics[width = \textwidth]{"../Files/Week 13/images/24_time"}
        \caption{Return times}
    \end{subfigure}
    ~ 
    \begin{subfigure}[t]{0.4\textwidth}
        \includegraphics[width=\textwidth]{"../Files/Week 13/images/24_mass"}
        \caption{Return masses}
    \end{subfigure}
    \begin{subfigure}[t]{0.4\textwidth}
        \includegraphics[width=\textwidth]{"../Files/Week 13/images/24_lyapunov"}
        \caption{Lyapunov exponent}
    \end{subfigure}
    \begin{subfigure}[t]{0.4\textwidth}
        \includegraphics[width=\textwidth]{"../Files/Week 13/images/24_ellipsoid"}
        \caption{Geometry}
    \end{subfigure}
    \caption{Distribution of the different properties for the galaxy with $a_1 = 1$, $a_2 = 6.6\times10^{-2}$, $a_3 = 6.1\times10^{-2}$.}
\end{figure}


%\include{Appendices/AppendixB}
%\include{Appendices/AppendixC}

%----------------------------------------------------------------------------------------
%	BIBLIOGRAPHY
%----------------------------------------------------------------------------------------

\printbibliography[heading=bibintoc, title={References}]

%----------------------------------------------------------------------------------------

\end{document}  
