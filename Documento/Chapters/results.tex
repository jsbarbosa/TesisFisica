% !TeX spellcheck = en_US
% Chapter 1

%\chapter{Chapter Title Here} % Main chapter title
%
%\label{Chapter1} % For referencing the chapter elsewhere, use \ref{Chapter1} 

%----------------------------------------------------------------------------------------

% Define some commands to keep the formatting separated from the content
%\newcommand{\option}[1]{\texttt{\itshape#1}}

%----------------------------------------------------------------------------------------
\chapter{Results}
	\section{Spherical study}
	For a single simulation, the following data is saved: iteration time, current position, speed, and the black hole mass. With these information, accelerations and densities can be later reconstructed as on \autoref{fig: overallOutput}.
	\begin{figure}[h]
		\centering
		\includegraphics[width = 0.52\textwidth]{"../Files/Week 6/properties_s02v70"}
		\caption{Upper two plots show the output of a single simulation, while the lower one shows most the local properties per data point.}
		\label{fig: overallOutput}
	\end{figure}
	
	\subsection{Effect of the baryonic fraction}
	The cosmological model used has two main effects in the simulations. The virial radius depends in the value of the Hubble parameter, that is different for different red-shifts and cosmological models (\autoref{fig: hubbleTime}), and the amount of matter in the universe. The effect of the later can be seen on \autoref{fig: baryonicfraction}, where results for a galaxy made by dark matter only allow the black hole to get further away, and dissipate energy slower.
	\begin{figure}[h]
		\centering
		\includegraphics[width = 0.7\textwidth]{"../Files/Week 5/baryonic_fraction_comparison"}
		\caption{Effect of the baryonic fraction in the orbit of the black hole.}
		\label{fig: baryonicfraction}
	\end{figure}

	Three factors need to be taken into account to explain these results. First, dynamical friction due to collisional matter has an important impact on the trajectories followed by the black hole, as can be seen with the green line of \autoref{fig: overallOutput}. Second, as there is no baryonic mass to accrete, the black hole does not decrease its speed because of momentum conservation, thus there is no contribution to the black hole's acceleration of the third term in \autoref{eq: equationMotion}. Lastly, as seen in \autoref{fig: massprofiles} at short distances, the cumulative mass of the galaxy is governed by stars, from which it is expected for the black hole to reach further regions of space if there are no stars, because of the smaller potential.
		
	\subsection{Effect of the power law exponent}	
		Many studied galaxies have luminosity profiles that follow a power law in distance. Nevertheless, power law density profiles have the exponent $n$ as free parameter, although it is expected to be approximate to 2, as rotation curves of galaxies at large radii, show that the rotation speed of stars becomes independent of their distance to the center \cite{binney2011galactic}. This can be easily shown by considering a system in which density drops with radius as a function of a $n$ power:
		\begin{equation}
			\rho(r) = \rho_0^\text{gas}\left(\dfrac{r_0}{r}\right)^n
		\end{equation}
		
		Hydrostatic equilibrium between the outward pressure of the galactic gas and the inward gravity from an spherical galaxy can be written as:
		\begin{equation}\label{eq: hydrostatic equilibrium}
			\dfrac{dp(r)}{dr} = -\rho(r)\dfrac{GM(r)}{r^2} = -\left[\rho_0^\text{gas}\left(\dfrac{r_0}{r}\right)^n\right]\dfrac{GM(r)}{r^2} = -\rho_0^\text{gas}r_0^nGM(r)r^{2-n}
		\end{equation}
		
		Stationary case comes if $n = 2$, as $dp(r) / dr = 0$, whenever this happens, pressure becomes independent of $r$. Nevertheless, as \autoref{eq: hydrostatic equilibrium} reveals, the value of $n$ strictly depends in the force generated by gravity, which in turn depends in the mass distribution within the galaxy. By taking the gradient of the potentials on \autoref{tb: potentials}, one can see that the NFW and Herquist profiles, generate forces almost but not equally proportional to $r^{-2}$.
		
		Moreover, many galaxies show a smooth transition between two distinct power laws, one for small radii, and another one for large distances \cite{binney2011galactic}. The general form for densities following a double power law is:
		\begin{equation}
			\rho(r) = \dfrac{\rho_0}{\left(\dfrac{r}{a}\right) ^ \alpha \left(1 + \dfrac{r}{a}\right) ^ {\beta - \alpha}}
		\end{equation}
		
		From which, one obtains all of the densities profiles used, as NFW is a double power law with $(\alpha, \beta) = (1, 3)$, Hernquist $(1, 4)$ and for gasses, $(0, n)$, as seen in equations \ref{eq: dmdensity}, \ref{eq: sdensity} and \ref{eq: rdensity}. Double power laws have an advantage over piecewise-defined functions, as they are smooth for all distances. 
		\begin{figure}[h]
			\centering
			\begin{subfigure}[t]{0.49\textwidth}
				\includegraphics[width = \textwidth]{"../Files/Week 6/power_law"}
				\caption{Effect of the power law exponent on the orbits of the black hole.}
				\label{fig: powerLawOrbits}
			\end{subfigure}
			~ 
			\begin{subfigure}[t]{0.49\textwidth}
				\includegraphics[width=\textwidth]{"../Files/Week 6/power_law_density"}
				\caption{Density and mass of the host galaxy as a function of the distance from the center, for different exponents.}
				\label{fig: powerLawDensities}
			\end{subfigure}
			\caption{Properties of the power law exponent.}
			\label{fig: powerLaw}
		\end{figure}
	
		Since $n$ depends , simulations were made for a range of exponents in the power law. Results can be seen in \autoref{fig: powerLaw}, were a clear influence of the exponent can be seen
	
	\subsection{Effect of the stellar fraction}
	\begin{figure}[h]
		\centering
		\includegraphics[width = 0.7\textwidth]{"../Files/Week 10/returntimes_stellar_speed"}
		\caption{Return times of the black hole for different initial speeds and stellar fractions.}
		\label{fig: stellarfraction}
	\end{figure}

	\begin{figure}[h]
		\centering
		\includegraphics[width = 0.9\linewidth]{"../Files/Week 9/PhaseSpace_escape"}
		\caption{Phase space generated with different stellar fractions, for an initial velocity $\vec{v} = 90\hat{i}$ kpc/Gyr.}
		\label{fig: escapePhaseSpace}
	\end{figure}
	\begin{figure}[h]
		\centering
		\includegraphics[width = 0.9\linewidth]{"../Files/Week 9/PhaseSpace_in"}
		\caption{Phase space generated with different stellar fractions, for an initial velocity $\vec{v} = 60\hat{i}$ kpc/Gyr.}
		\label{fig: escapeInner}
	\end{figure}

	\section{Triaxial study}
	\begin{figure}[h]
		\centering
		\begin{subfigure}[b]{0.49\textwidth}
			\includegraphics[width = \textwidth]{"../Files/Week 7/orthogonal_triaxial"}
			\caption{Orbits for three orthogonal launches}
			\label{fig: orthogonalLaunches}
		\end{subfigure}
		~ 
		\begin{subfigure}[b]{0.49\textwidth}
			\includegraphics[width=\textwidth]{"../Files/Week 7/ellipsoid"}
			\caption{Elliptical geometry}
		\end{subfigure}
		\caption{Orthogonal launches for a triaxial profile with semi-axis ($a_1$:$a_2$:$a_3$) = (1:0.99:0.95)}
		\label{fig: mainOrthogonalLaunches}
	\end{figure}