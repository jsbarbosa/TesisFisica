% !TeX spellcheck = en_US
% Chapter 1

%\chapter{Chapter Title Here} % Main chapter title
%
%\label{Chapter1} % For referencing the chapter elsewhere, use \ref{Chapter1} 

%----------------------------------------------------------------------------------------

% Define some commands to keep the formatting separated from the content
%\newcommand{\option}[1]{\texttt{\itshape#1}}

%----------------------------------------------------------------------------------------
\chapter{Results}
	\section{Spherical study}
	For a single simulation, the following data is saved: iteration time, current position, speed, and the black hole mass. With these information, accelerations and densities can be later reconstructed as on \autoref{fig: overallOutput}.
	\begin{figure}[h]
		\centering
		\includegraphics[width = 0.52\textwidth]{"../Files/Week 6/properties_s02v70"}
		\caption{Upper two plots show the output of a single simulation, while the lower one shows most the local properties per data point.}
		\label{fig: overallOutput}
	\end{figure}
	
	\subsection{Effect of the baryonic fraction}
	The cosmological model used has two main effects in the simulations. The virial radius depends in the value of the Hubble parameter, that is different for different red-shifts and cosmological models (\autoref{fig: hubbleTime}), and the amount of matter in the universe. The effect of the later can be seen on \autoref{fig: baryonicfraction}, where results for a galaxy made by dark matter only allow the black hole to get further away, and dissipate energy slower.
	\begin{figure}[h]
		\centering
		\includegraphics[width = 0.7\textwidth]{"../Files/Week 5/baryonic_fraction_comparison"}
		\caption{Effect of the baryonic fraction in the orbit of the black hole.}
		\label{fig: baryonicfraction}
	\end{figure}

	Three factors need to be taken into account to explain these results. First, dynamical friction due to collisional matter has an important impact on the trajectories followed by the black hole, as can be seen with the green line of \autoref{fig: overallOutput}. Second, as there is no baryonic mass to accrete, the black hole does not decrease its speed because of momentum conservation, thus there is no contribution to the black hole's acceleration of the third term in \autoref{eq: equationMotion}. Lastly, as seen in \autoref{fig: massprofiles} at short distances, the cumulative mass of the galaxy is governed by stars, from which it is expected for the black hole to reach further regions of space if there are no stars, because of the smaller potential.
		
	\subsection{Effect of the power law exponent}
		Gasses 
		\begin{figure}[h]
			\centering
			\begin{subfigure}[t]{0.49\textwidth}
				\includegraphics[width = \textwidth]{"../Files/Week 6/power_law"}
				\caption{Effect of the power law exponent on the orbits of the black hole.}
				\label{fig: powerLawOrbits}
			\end{subfigure}
			~ 
			\begin{subfigure}[t]{0.49\textwidth}
				\includegraphics[width=\textwidth]{"../Files/Week 6/power_law_density"}
				\caption{Density and mass of the host galaxy as a function of the distance from the center, for different exponents.}
				\label{fig: powerLawDensities}
			\end{subfigure}
			\caption{Properties of the power law exponent.}
			\label{fig: powerLaw}
		\end{figure}
	
	\subsection{Effect of the stellar fraction}
	\begin{figure}[h]
		\centering
		\includegraphics[width = 0.7\textwidth]{"../Files/Week 10/returntimes_stellar_speed"}
		\caption{Return times of the black hole for different initial speeds and stellar fractions.}
		\label{fig: stellarfraction}
	\end{figure}

	\begin{figure}[h]
		\centering
		\includegraphics[width = 0.9\linewidth]{"../Files/Week 9/PhaseSpace_escape"}
		\caption{Phase space generated with different stellar fractions, for an initial velocity $\vec{v} = 90\hat{i}$ kpc/Gyr.}
		\label{fig: escapePhaseSpace}
	\end{figure}
	\begin{figure}[h]
		\centering
		\includegraphics[width = 0.9\linewidth]{"../Files/Week 9/PhaseSpace_in"}
		\caption{Phase space generated with different stellar fractions, for an initial velocity $\vec{v} = 60\hat{i}$ kpc/Gyr.}
		\label{fig: escapeInner}
	\end{figure}

	\section{Triaxial study}
	\begin{figure}[h]
		\centering
		\begin{subfigure}[b]{0.49\textwidth}
			\includegraphics[width = \textwidth]{"../Files/Week 7/orthogonal_triaxial"}
			\caption{Orbits for three orthogonal launches}
			\label{fig: orthogonalLaunches}
		\end{subfigure}
		~ 
		\begin{subfigure}[b]{0.49\textwidth}
			\includegraphics[width=\textwidth]{"../Files/Week 7/ellipsoid"}
			\caption{Elliptical geometry}
		\end{subfigure}
		\caption{Orthogonal launches for a triaxial profile with semi-axis ($a_1$:$a_2$:$a_3$) = (1:0.99:0.95)}
		\label{fig: mainOrthogonalLaunches}
	\end{figure}