% !TeX spellcheck = en_US
% Chapter 1

%\chapter{Chapter Title Here} % Main chapter title
%
%\label{Chapter1} % For referencing the chapter elsewhere, use \ref{Chapter1} 

%----------------------------------------------------------------------------------------

% Define some commands to keep the formatting separated from the content
%\newcommand{\option}[1]{\texttt{\itshape#1}}

%----------------------------------------------------------------------------------------
\chapter{Results}
	\section{Spherical study}
	For a single simulation, the following data is saved: iteration time, current position, speed, and the black hole mass. With these information, accelerations and densities can be later reconstructed as on \autoref{fig: overallOutput}.
	\begin{figure}[h]
		\centering
		\includegraphics[width = 0.52\textwidth]{"../Files/Week 6/properties_s02v70"}
		\caption{Upper two plots show the output of a single simulation, while the lower one shows most the local properties per data point.}
		\label{fig: overallOutput}
	\end{figure}
	
	\subsection{Effect of the baryonic fraction}
	\begin{figure}[h]
		\centering
		\includegraphics[width = 0.7\textwidth]{"../Files/Week 5/baryonic_fraction_comparison"}
		\caption{Effect of the baryonic fraction in the orbit of the black hole.}
		\label{fig: baryonicfraction}
	\end{figure}
		
	\subsection{Effect of the power law exponent}
	\begin{figure}[h]
		\centering
		\begin{subfigure}[t]{0.49\textwidth}
			\includegraphics[width = \textwidth]{"../Files/Week 6/power_law"}
			\caption{Effect of the power law exponent on the orbits of the black hole.}
			\label{fig: powerLawOrbits}
		\end{subfigure}
		~ 
		\begin{subfigure}[t]{0.49\textwidth}
			\includegraphics[width=\textwidth]{"../Files/Week 6/power_law_density"}
			\caption{Density and mass of the host galaxy as a function of the distance from the center, for different exponents.}
			\label{fig: powerLawDensities}
		\end{subfigure}
		\caption{Properties of the power law exponent.}
		\label{fig: powerLaw}
	\end{figure}
	
	\subsection{Effect of the stellar fraction}
	\begin{figure}[h]
		\centering
		\includegraphics[width = 0.7\textwidth]{"../Files/Week 7/Symmetric/returntimes_stellar_speed"}
		\caption{Return times of the black hole for different initial speeds and stellar fractions.}
		\label{fig: stellarfraction}
	\end{figure}

	\begin{figure}[h]
		\centering
		\includegraphics[width = 0.7\textwidth]{"../Files/Week 7/Symmetric/returntimes_mass"}
		\caption{Return masses of the black hole for different initial speeds and stellar fractions as a function of their return time.}
		\label{fig: returnMass}
	\end{figure}

	\begin{figure}[h]
		\centering
		\includegraphics[width = 0.9\linewidth]{"../Files/Week 9/PhaseSpace_escape"}
		\caption{Phase space generated with different stellar fractions, for an initial velocity $\vec{v} = 90\hat{i}$ kpc/Gyr.}
		\label{fig: escapePhaseSpace}
	\end{figure}
	\begin{figure}[h]
		\centering
		\includegraphics[width = 0.9\linewidth]{"../Files/Week 9/PhaseSpace_in"}
		\caption{Phase space generated with different stellar fractions, for an initial velocity $\vec{v} = 60\hat{i}$ kpc/Gyr.}
		\label{fig: escapeInner}
	\end{figure}

	\section{Triaxial study}
	
	\begin{figure}[h]
		\centering
		\begin{subfigure}[b]{0.49\textwidth}
			\includegraphics[width = \textwidth]{"../Files/Week 7/orthogonal_triaxial"}
			\caption{Orbits for three orthogonal launches}
			\label{fig: orthogonalLaunches}
		\end{subfigure}
		~ 
		\begin{subfigure}[b]{0.49\textwidth}
			\includegraphics[width=\textwidth]{"../Files/Week 7/ellipsoid"}
			\caption{Elliptical geometry}
		\end{subfigure}
		\caption{Orthogonal launches for a triaxial profile with semi-axis ($a_1$:$a_2$:$a_3$) = (1:0.99:0.95)}
		\label{fig: mainOrthogonalLaunches}
	\end{figure}