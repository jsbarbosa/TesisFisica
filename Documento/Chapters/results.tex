% !TeX encoding = UTF-8
% !TeX spellcheck = en_US
% Chapter 1

%\chapter{Chapter Title Here} % Main chapter title
%
%\label{Chapter1} % For referencing the chapter elsewhere, use \ref{Chapter1} 

%----------------------------------------------------------------------------------------

% Define some commands to keep the formatting separated from the content
%\newcommand{\option}[1]{\texttt{\itshape#1}}

%----------------------------------------------------------------------------------------
\chapter{Results and discussion}
	\section{Spherical study}
	For a single simulation, the following data is saved: iteration time, current position, speed, and the black hole mass. With these information, accelerations and densities can be later reconstructed as on \autoref{fig: overallOutput}.
	\begin{figure}[h]
		\centering
		\includegraphics[width = 0.52\textwidth]{"../Files/Week 6/properties_s02v70"}
		\caption{Upper two plots show the output of a single simulation, while the lower one shows most the local properties per data point.}
		\label{fig: overallOutput}
	\end{figure}
	
	\subsection{Effect of the baryonic fraction}
	The cosmological model used has two main effects in the simulations. The virial radius depends on the value of the Hubble parameter, that changes for different red-shifts and cosmological models (\autoref{fig: hubbleTime}), and the amount of matter in the universe. This effect can be seen on \autoref{fig: baryonicfraction}, where results for a galaxy made by dark matter only allow the black hole to get further away, and dissipate energy slower.
	\begin{figure}[h]
		\centering
		\includegraphics[width = 0.7\textwidth]{"../Files/Week 5/baryonic_fraction_comparison"}
		\caption{Effect of the baryonic fraction in the orbit of the black hole.}
		\label{fig: baryonicfraction}
	\end{figure}

	Three factors need to be taken into account to explain these results. First, dynamical friction due to collisional matter has an important impact on the trajectories followed by the black hole, as can be seen with the green line of \autoref{fig: overallOutput} ($a_\text{DF}$). Second, as there is no baryonic mass to accrete, the black hole does not decrease its speed because of momentum conservation, thus there is no contribution to the black hole's acceleration of the third term in \autoref{eq: equationMotion}. Lastly, as seen in \autoref{fig: massprofiles} at short distances, the cumulative mass of the galaxy is governed by stars, from which it is expected for the black hole to reach further regions of space if there are no stars, because of the smaller potential.
		
	\subsection{Effect of the power law exponent}	
		Many studied galaxies have luminosity profiles that follow a power law in distance. Nevertheless, power law density profiles have the exponent $n$ as free parameter, although it is expected to be approximate to 2, as rotation curves of galaxies at large radius, show that the rotation speed of stars becomes independent of their distance to the center \cite{binney2011galactic}. This can be shown by considering a system in which density drops with radius as a function of a $n$ power:
		\begin{equation}
			\rho(r) = \rho_0^\text{gas}\left(\dfrac{r_0}{r}\right)^n
		\end{equation}
		
		Hydrostatic equilibrium between the outward pressure of the galactic gas and the inward gravity from an spherical galaxy can be written as:
		\begin{equation}\label{eq: hydrostatic equilibrium}
			\dfrac{dp(r)}{dr} = -\rho(r)\dfrac{GM(r)}{r^2} = -\left[\rho_0^\text{gas}\left(\dfrac{r_0}{r}\right)^n\right]\dfrac{GM(r)}{r^2} = -\rho_0^\text{gas}r_0^nGM(r)r^{2-n}
		\end{equation}
		
		Stationary case comes if $n = 2$, as $dp(r) / dr = 0$ whenever this happens, pressure becomes independent of $r$. Nevertheless, as \autoref{eq: hydrostatic equilibrium} reveals, the value of $n$ strictly depends on the force generated by gravity, which in turn depends on the mass distribution within the galaxy. By taking the gradient of the potentials on \autoref{tb: potentials}, one can see that the NFW and Herquist profiles generate forces almost but not equally proportional to $r^{-2}$.
		
		Moreover, many galaxies show a smooth transition between two distinct power laws, one for small radii, and another one for large distances \cite{binney2011galactic}. The general form for densities following a double power law is:
		\begin{equation}
			\rho(r) = \dfrac{\rho_0}{\left(\dfrac{r}{a}\right) ^ \alpha \left(1 + \dfrac{r}{a}\right) ^ {\beta - \alpha}}
		\end{equation}
		
		From which one obtains all of the densities profiles used, as NFW is a double power law with $(\alpha, \beta) = (1, 3)$, Hernquist $(1, 4)$ and for gases, $(0, n)$, as seen in equations \ref{eq: dmdensity}, \ref{eq: sdensity} and \ref{eq: rdensity}. Double power laws have an advantage over piecewise-defined functions, as they are smooth for all distances. 		
		\begin{figure}[h]
			\centering
			\begin{subfigure}[t]{0.49\textwidth}
				\includegraphics[width = \textwidth]{"../Files/Week 6/power_law"}
				\caption{Effect of the power law exponent on the orbits of the black hole.}
				\label{fig: powerLawOrbits}
			\end{subfigure}
			~ 
			\begin{subfigure}[t]{0.49\textwidth}
				\includegraphics[width=\textwidth]{"../Files/Week 6/power_law_density"}
				\caption{Density and mass of the host galaxy as a function of the distance from the center, for different exponents.}
				\label{fig: powerLawDensities}
			\end{subfigure}
			\caption{Properties of the power law exponent.}
			\label{fig: powerLaw}
		\end{figure}
	
		Since $n$ is not fixed, simulations were made for a range of exponents in the power law. Results can be seen in \autoref{fig: powerLaw}, were a clear influence of the exponent can be seen. A confirmation that everything went the way it is supposed to, is that cumulative masses for all of the exponents have the same value at the virial radius (red line in \autoref{fig: powerLawDensities}). At this same distance, the behavior of the mass for each exponent changes, as smaller values of $n$ have a bigger cumulative masses from there on. At any other point, the density for bigger values of $n$ is greater, which means that both dynamical friction and accretion rates increase for the black hole as $n$ get bigger. Also, as $n$ becomes greater, cumulative masses increase, yielding a higher gravitational potential. Both of these effects, take part in the observed results in \autoref{fig: powerLawOrbits}, in which, higher values of $n$ increase return times.
		
		There are some spiral galaxies from which an exponent of 2.6 has been calculated as in NGC 253 \cite{sorai2000distribution} others authors have made simulations for black hole escape velocities with $n = 2.2$ \cite{tanaka2009assembly, choksi2017recoiling}. It is from these references, from which the value of $n = 2.2$ for most simulations was selected.
		
	\subsection{Effect of the stellar fraction}
		As was mentioned before, the mass of a galaxy at low distances from its center, is ruled by the amount of stars. Because of this, a study of the dependence of the return times with stellar fraction, and initial speed was carried on. For that, simulations with initial speeds from 55 to 90 kpc/Gyr were lunched, for stellar fractions ranging from 1 \% to 10 \%. In order to normalize speeds, initial speeds are divided by the escape velocity. By considering the potential energy at the edge of the galaxy, and the initial energy of the black hole, the escape velocity is written as:
		\begin{equation}
			v_\text{escape} = \sqrt{2 \left(\Phi(R_\text{vir}) - \Phi(r_0)\right)}
		\end{equation}
		
		Results from the simulations can be seen on \autoref{fig: stellarfraction}, where a linear behavior below the escape speed is common for all stellar fractions. As the initial velocities approximate 1.3, an exponential growth is seen, and return times become divergent.
		\begin{figure}[h]
			\centering
			\includegraphics[width = 0.7\textwidth]{"../Files/Week 10/returntimes_speed"}
			\caption{Return times of the black hole for different initial speeds and stellar fractions.}
			\label{fig: stellarfraction}
		\end{figure}
	
		Return times are thus fitted to the following equation:
		\begin{equation}\label{eq: fitTr}
			\log_{10}(T_\text{return}) = [a(f_s) v + b(f_s)] + \dfrac{c(f_s)}{v - 1.3}
		\end{equation}
		
		Where the first term accounts for the linear behavior and the last for the divergence at high velocities. Using the information in \autoref{fig: stellarfraction}, coefficients in \autoref{eq: fitTr} are calculated for each stellar fraction. Later, with the value of the coefficients, a second fit is made for $a(f_s), b(f_s), c(f_s)$.
		
		\begin{figure}[h]
			\centering
			\begin{subfigure}[t]{0.4\textwidth}
				\includegraphics[width = \textwidth]{"../Files/Week 10/a"}
				\caption{}
%				\label{fig: symmetricDensity3d}
			\end{subfigure}
			~ 
			\begin{subfigure}[t]{0.4\textwidth}
				\includegraphics[width=\textwidth]{"../Files/Week 10/b"}
				\caption{}
%				\label{fig: triaxialDensity3d}
			\end{subfigure}
			\begin{subfigure}[t]{0.4\textwidth}
				\includegraphics[width=\textwidth]{"../Files/Week 10/c"}
				\caption{}
			\end{subfigure}
			\caption{Fits for the coefficients in \autoref{eq: fitTr}}
			\label{fig: coeffsFits}
		\end{figure}

	Using \autoref{eq: fitTr} and the fits in \autoref{fig: coeffsFits}, the whole surface of return times is reconstructed, enabling the possibility of semianalytical calculations without the need of simulations, as the dependence of the coefficients with the stellar fraction is:
	\begin{equation}
		a(f_s) = 232f_s^2 + 25 f_s + 2.83
	\end{equation} 
	\begin{equation}
		b(f_s) = -40.7 f_s - 0.75
	\end{equation}
	\begin{equation}
		c(f_s) = 60 f_s^2 - 2.8 f_s - 0.080
	\end{equation}

	\begin{figure}[h]
		\centering
		\includegraphics[width = 0.7\textwidth]{"../Files/Week 10/surface"}
		\caption{Constructed surface of return times of the black hole for different initial speeds and stellar fractions.}
		\label{fig: stellarfraction3d}
	\end{figure}

	Nevertheless, as the constructed surface in \autoref{fig: stellarfraction3d} shows, predicted return times for both, high stellar fractions and initial speeds, are smaller than expected.
	
	\begin{figure}[h!]
		\centering
		\includegraphics[width = 0.9\linewidth]{"../Files/Week 9/PhaseSpace_escape"}
		\caption{Phase space generated with different stellar fractions, for an initial velocity $\vec{v} = 90\hat{i}$ kpc/Gyr.}
		\label{fig: escapePhaseSpace}
	\end{figure}

	On the other hand, phase spaces for high and low initial speeds, show how the increase in stellar fraction make it harder for the black hole to get further in space. In both cases, initial velocity is only in the $x$ direction, thus, curves in $y$ and $z$ happen because the initial position is not exactly $\vec{0}$, but $(1, 1, 1)$ pc, that also explains why the behavior in $y$ an $z$ is identical to one another.
	
	\begin{figure}[h]
		\centering
		\includegraphics[width = 0.9\linewidth]{"../Files/Week 9/PhaseSpace_in"}
		\caption{Phase space generated with different stellar fractions, for an initial velocity $\vec{v} = 60\hat{i}$ kpc/Gyr.}
		\label{fig: escapeInner}
	\end{figure}

	An interesting behavior in \autoref{fig: escapePhaseSpace} is that for the 3 smaller stellar fractions, the simulated black holes are still increasing their distance from their host galaxy as of today, while the fourth curve shows a black hole in which gravity has already overcome the initial speed, and changed the direction of the velocity.
	
	\section{Triaxial study}
	By using the information in \autoref{fig: stellarfraction} relative initial speeds for the simulated galaxies were generated randomly with magnitudes ranging from $|\vec{v_0}/v_{\text{escape}}| = 0.7$ to $|\vec{v_0}/v_{\text{escape}}| = 1.2$, with random directions of the speed within the positive defined eighth of a sphere, as in \autoref{fig: initialSpeedDistributions}.
	
	\begin{figure}[h]
		\centering
		\begin{subfigure}[b]{0.49\textwidth}
			\includegraphics[width = \textwidth]{"../Files/Week 13/3d_initial_speeds"}
			\caption{Cartesian}
%			\label{fig: orthogonalLaunches}
		\end{subfigure}
		~ 
		\begin{subfigure}[b]{0.49\textwidth}
			\includegraphics[width=\textwidth]{"../Files/Week 13/polar_initial_speeds"}
			\caption{Polar}
		\end{subfigure}
		\caption{Distributions of initial speeds for the triaxial lunches. $\theta$ describes the polar angle and $\phi$ the azimuth.}
		\label{fig: initialSpeedDistributions}
	\end{figure}

	A total of 28 different pairs of semiaxis $(a_2, a_3)$ were generated randomly using an uniform distribution. A condition was imposed for $a_3$ to be smaller or equal to $a_2$. The distribution of the semiaxis pairs can be seen in \autoref{fig: semiaxisDist}. In order to describe how close a generated galaxy is to an spherical reference, the Triaxial parameter ($T$) is calculated.
	\begin{equation}\label{eq: triaxialParameter}
		T(a_1, a_2, a_3) = \dfrac{1 - \left(\dfrac{a_2}{a_1}\right)^2}{1 - \left(\dfrac{a_3}{a_1}\right)^2}
	\end{equation}
	
	Nevertheless, caution must be taken as two galaxies with the exact same $T$ may look completely different as there are two degrees of freedom in \autoref{eq: triaxialParameter}. Since all values of $a_1 = 1$, the total degrees of freedom decreases to one, meaning that one pair of values of $(a_2, a_3)$ might yield the same triaxial parameter as another pair $(a_2', a_3')$.
	
	\begin{figure}[h]
		\centering
		\includegraphics[width = 0.9\linewidth]{"../Files/Week 13/triaxial_axes"}
		\caption{Distribution of the 28 values for the $y$ and $z$ semiaxis.}
		\label{fig: semiaxisDist}
	\end{figure}

	\subsection{Return distributions}
	One of the results from the 28,000 simulations is shown in \autoref{fig: massDist}, where the probability density for the return masses is plotted. Since results come from 28 different galaxies geometries, with triaxial parameters in the range of $9.5\times10^{-3}$, and 1.0, the probability densities for the return properties are independent of the galactic shapes, in fact they are considered to be an interesting statistical measurement of the overall behavior expected in the universe.
	\begin{figure}[h]
		\centering
		\includegraphics[width=0.7\linewidth]{"../Files/Week 14/dist_masses"}
		\caption{Mass distributions of the returned black hole, for the 28 triaxial lunches (blue) and for an spherical galaxy.}
		\label{fig: massDist}
	\end{figure}

	\autoref{fig: massDist} shows that the mass at the return time follows a gaussian Probability Density Function (PDF) given by \autoref{eq: gaussian}, where $\sigma^2$ is variance of the data, and $\mu$ the mean.
	\begin{equation}\label{eq: gaussian}
		\text{PDF}_\text{Gauss}(x, \mu, \sigma) = \dfrac{1}{\sqrt{2\pi\sigma^2}}e^{-\frac{(x - \mu)^2}{2\sigma^2}}
	\end{equation}
	
	Both the mean and the variance of the fitted gaussians for the distribution of return masses are shown in \autoref{tb: gaussians} for comparison, the distribution of masses for a spherical galaxy with the same initial speeds is shown.
	\begin{table}[h]
		\centering
		\caption{Parameters of the fitted gaussians in \autoref{fig: massDist}.}
		\begin{tabular}{r|cc}
			\hline
			& \textbf{Spherical} & \textbf{Triaxial} \\
			\hline
			$\mu$ ($10^5$ \sm)& 1.08 & 1.34 \\
			$\sigma^2$ ($10^5$ \sm)$^2$& $3.7\times10^{-4}$ & $1.7\times10^{-2}$\\
			\hline
		\end{tabular}
		\label{tb: gaussians}
	\end{table}

	These results are in concordance with previous studies, where authors argue that black holes need to be at high density areas of the galaxy, such as its center, in order to radically increase its mass by accreting material from the surroundings. One of the consequences from mass accretion is the possibility for the black hole to eventually become a quasar \cite{tanaka2009assembly}, \autoref{tb: gaussians} shows that on average, while a black hole experiences a kick, its mass increases only by a factor of 1.34. Quasars have masses of the order of $10^8$ \sm, thus it is highly difficult for a kicked black hole to become a quasar while experiencing the actual kick.
	
	\begin{figure}[h]
		\centering
		\begin{subfigure}[b]{0.49\textwidth}
			\includegraphics[width = \textwidth]{"../Files/Week 14/mass_behavior"}
			\caption{Mass in time.}
			\label{fig: distanceMass}
		\end{subfigure}
		~ 
		\begin{subfigure}[b]{0.49\textwidth}
			\includegraphics[width=\textwidth]{"../Files/Week 14/mass_accretion"}
			\caption{Accretion in time.}
			\label{fig: distanceAccretion}
		\end{subfigure}
		\caption{Distance of the black hole (black), and mass information in blue, with respect to time.}
		\label{fig: accretionEff}
	\end{figure}
	
	Whenever the black hole is close to its host galaxy center its mass increases rapidly, (\autoref{fig: distanceMass}). By differentiating in time, accretion rate is found, \autoref{fig: distanceAccretion} shows both the Eddington accretion and the actual simulated accretion. There, the capping effect of \autoref{eq: finalAccretion} can be seen, particularly at long times. Furthermore, as the area under the curve yields the mass of the black hole, all of the orange shaded region does not contribute to growing of a kicked black hole, while it would for an static black hole.
	
	\begin{figure}[h]
		\centering
		\includegraphics[width = 0.8\linewidth]{"../Files/Week 14/dist_times"}
		\caption{Return time distributions, for the 28 triaxial lunches (blue) and for an spherical galaxy. Below, the cumulative probability of the PDFs.}
		\label{fig: timeDist}
	\end{figure}

	As for return times, they have a much wider distribution, as seen in \autoref{fig: timeDist} where the data follows a logarithmic distribution. Just as with the masses, fully triaxial galaxies have higher return times when compared to an spherical galaxy. The distributions of return times are fitted to the superposition of two gaussians as described in \autoref{eq: timeDist}.
	\begin{equation}\label{eq: timeDist}
		\text{PDF}(t) = \alpha \text{PDF}_\text{Gauss}(\log_{10}t, \mu_1, \sigma_1) + \beta \text{PDF}_\text{Gauss}(\log_{10}t, \mu_2, \sigma_2)
	\end{equation}
	
	\begin{table}[h]
		\centering
		\caption{Fitted values for the parameters in \autoref{eq: timeDist}, by using \autoref{fig: timeDist}.}
		\begin{tabular}{r|cc}
			\hline
			& \textbf{Spherical} & \textbf{Triaxial} \\
			\hline
			$\mu_1$ $\log$(Myr) & 2.22 & 2.60\\
			$\sigma_1$ $\log$(Myr) & 0.81 & 0.64\\
			$\mu_2$ $\log$(Myr) & 1.36 & 1.81\\
			$\sigma_1$ $\log$(Myr) & 0.27 & 0.23\\
			$\alpha$ & $4.5\times10^{-4}$ & $3.5\times10^{-4}$\\
			$\beta$ & $1.1\times10^{-4}$ & $1.2\times10^{-4}$\\
			\hline
		\end{tabular}
	\end{table}
	
	Additionally, with the information from the simulations the probability of finding a quasar (like ULAS J1342+0928) generated from a kicked black hole can be estimated. ULAS J1342+0928 is the farthest known quasar, it has and approximate mass of $8\times10^8$ \sm, and it is found at redshift $Z = 7.54$ \cite{banados2018800, paris2018sloan}. By considering the $\Lambda$-CDM cosmological model with the parameters described in the \autoref{ch: methodology}, redshifts can be converted to time, for instance, redshift $Z = 7.54$ is about 692 Myr from the start of the universe, while $Z = 20$ (the redshift at which the studied black holes get a kick) is 180 Myr. This means that by evaluating the mass of all the 28,000 simulated black holes at $692 - 180 = 512$ Myr, the probability of finding a black hole with a mass of $8\times10^8$ \sm can be calculated from the distribution of the simulated black holes masses. Such distribution can be seen in \autoref{fig: massDistAt}, where the fitted density function is:
	\begin{equation}
		\text{PDF}_{t = 512} = 10 ^ {a * \log_{10}m + b}
	\end{equation}
	
	where $a = -0.75 \pm 0.03$ $\log_{10}^{-1}M_\odot$, and $b = 3.8 \pm 0.2$. By integrating this equation from $10^8$ to $10^9$ \sm, the probability for finding a quasar with such masses from the simulation is calculated in 0.34 \%.
	\begin{figure}[h]
		\centering
		\includegraphics[width=0.8\linewidth]{"../Files/Week 14/masses_at"}
		\caption{Mass distributions at $t = 512$ Myr. Red line shows the mass value from an static black hole with the same seed mass experiencing Eddington accretion for 512 Myr.}
		\label{fig: massDistAt}
	\end{figure}

	\autoref{fig: massDistAt}, shows how the probability falls quickly with the mass, this result explains why finding a quasar such as ULAS J1342+0928 at high redshift is difficult, as there is not much time for the black hole to accrete much mass if the seed black hole has complicated dynamics such as kicks.
	
	\subsection{Correlation between mass and time}
	The spatial distribution of the return times, masses and Lyapunov exponents for each galaxy can be seen in the \autoref{ch: Galaxies}. By inspecting these results, a correlation between the return times and masses can be seen. By considering the Pearson moment of correlation, this behavior can be quantified using \autoref{eq: pearson}.
	\begin{equation}\label{eq: pearson}
		r_{\log_{10}t, m} = \dfrac{\text{cov}(\log_{10}t, m)}{\sigma_{\log_{10}t}\sigma_m}
	\end{equation}
	
	\begin{figure}[h]
		\centering
		\begin{subfigure}[b]{0.49\textwidth}
			\includegraphics[width = \textwidth]{"../Files/Week 13/correlation_T"}
			\caption{Colors identify the triaxial parameter of each galaxy.}
			\label{fig: triaxial_correlation}
		\end{subfigure}
		~ 
		\begin{subfigure}[b]{0.49\textwidth}
			\includegraphics[width = \textwidth]{"../Files/Week 13/correlation_speed"}
			\caption{Colors identify the initial relative speed of a specific return time and mass.}
		\end{subfigure}
		\caption{Correlation between the return times and the return masses.}
		\label{fig: speed_correlation}
	\end{figure}
	
	The calculated value, for all the studied orbits, of the Pearson correlation is 0.63, which is in accordance with the behavior seen in both the \autoref{ch: Galaxies} and \autoref{fig: speedDist}, describing a positive correlation between the masses and return times, that is, that and increase in the return mass most of the times will mean that the return time also increases. Furthermore, the triaxial parameter seems to change the slope in \autoref{fig: triaxial_correlation}, smaller $T$ galaxies have a tendency to have higher slopes than galaxies with triaxial parameters close to the unity. Additionally, when the correlation is studied with respect to the initial speed of each black hole, horizontal zones are found, this means that the effect of the initial speed in the return mass is far less than the effect in the return time, allow the formation of colored zones in \autoref{fig: massSpeedDist}. 
	
	\subsection{Triaxial vs Spherical galaxies}
	An important consequence of triaxial galaxies is that the magnitude of the velocity can not longer predict a unique return time, since there is a complete probability distribution associated to one value of the speed. These can clearly be seen in \autoref{fig: timeSpeedDist}, where a graph analogous to \autoref{fig: stellarfraction} for the galaxies with lower and higher values of the triaxial parameter. These results are particularly important as for a single speed, depending on the direction, two very different (almost one order of magnitude) return times will be generated for the same triaxial galaxy. This same figure shows that depending on the direction of the velocity, return times for triaxial galaxies might be shorter than thus expected with using an spherical model.
	\begin{figure}[h]
		\centering
		\begin{subfigure}[t]{0.49\textwidth}
			\includegraphics[width = \linewidth]{"../Files/Week 13/rt_speed"}
			\caption{Return times.}
			\label{fig: timeSpeedDist}
		\end{subfigure}
		~ 
		\begin{subfigure}[t]{0.49\textwidth}
			\includegraphics[width = \linewidth]{"../Files/Week 13/rt_mass"}
			\caption{Return masses.}
			\label{fig: massSpeedDist}
		\end{subfigure}
		\caption{Probability distributions for the galaxies with higher and lower (red and blue) triaxial parameters, for the return properties as a function of the initial speed.}
		\label{fig: speedDist}
	\end{figure}
	
	As for return masses, triaxial galaxies consistently show higher values than those generated by the same initial conditions in an spherical galaxy. Nevertheless, all galaxies have a small tendency to increase its return mass with increasing initial speeds. On the other hand, caution must be taken with \autoref{fig: massSpeedDist}, as mass distributions seem to be much wider than those of the return times, that is just a visual effect due to the logarithmic axis of \autoref{fig: timeSpeedDist}. 
	
	Despite the fact that \autoref{fig: speedDist} only shows results for two triaxial galaxies, an interesting behavior is found when comparing these results with those from an spherical galaxy. For small initial speeds, triaxial galaxies show higher return times than their spherical analog, while for high initial speeds this pattern gets inverted for big $T$ galaxies (red curves in \autoref{fig: timeSpeedDist}, red dots in \autoref{fig: relTime}). To study further the effect of circularity in the predicted times, the coefficient triaxial/spherical for the return properties in plotted in \autoref{fig: relatives}. In \autoref{fig: relTime}, however, the argument of a shift between the predictions of triaxial and spherical galaxies can be better seen, since on average relative return times follow a positive slope with respect to higher initial speeds, nevertheless, this also increases uncertainty.
	\begin{figure}[h!]
		\centering
		\begin{subfigure}[b]{0.49\textwidth}
			\includegraphics[width = \textwidth]{"../Files/Week 14/relative_times"}
			\caption{Relative return times as a function of initial speed.}
			\label{fig: relTime}
		\end{subfigure}
		~ 
		\begin{subfigure}[b]{0.49\textwidth}
			\includegraphics[width = \textwidth]{"../Files/Week 14/relative_mass"}
			\caption{Relative return masses as a function of initial speed.}
			\label{fig: relMass}
		\end{subfigure}
		\begin{subfigure}[b]{0.49\textwidth}
			\includegraphics[width = 0.9\textwidth]{"../Files/Week 14/relative_times_dist"}
			\caption{Relative return times distribution.}
			\label{fig: relTimeDist}
		\end{subfigure}
		~ 
		\begin{subfigure}[b]{0.49\textwidth}
			\includegraphics[width = 0.9\textwidth]{"../Files/Week 14/relative_mass_dist"}
			\caption{Relative return masses distribution.}
			\label{fig: relMassDist}
		\end{subfigure}
		\caption{Relative distributions of the return properties.}
		\label{fig: relatives}
	\end{figure}

	Return masses exhibit the same behavior as initial speeds increase, so does the difference of the predicted values of the return masses of triaxial galaxies with respect to their spherical analog, and uncertainty increases with speed. However, this changes are far smaller than those seen with the relative return times.
	
	Figures \ref{fig: relTimeDist} and \ref{fig: relMassDist} show the global distribution of the relative return properties. These are very important, as they graphically show how triaxiality affects previous studies of recoiling black holes orbits, such as \citeauthor{choksi2017recoiling}, \citeauthor{tanaka2009assembly} and \citeauthor{gualandris2008ejection}. Since return times for triaxial galaxies can be more than 10 times higher than expected, and galatic geometries in the universe are very diverse, triaxiality needs to be included when studying the dynamics of a recoiling black hole. On the other hand, triaxiality has a lower effect in the return masses, as on average 24 \% bigger masses are expected. Finally, statistical information from figures \ref{fig: relTimeDist} and \ref{fig: relMassDist} is summarized in \autoref{tb: relDist}.
	\begin{table}[h]
		\centering
		\caption{Mean ($\mu$) and standard deviation ($\sigma$) from relative return times and masses.}
		\begin{tabular}{r|cc}
			\hline & \textbf{Relative time} & \textbf{Relative mass} \\
			\hline
			$\mu$ & 2.61 & 1.24 \\
			$\sigma$ & 1.34 & 0.12 \\
			\hline
		\end{tabular}
		\label{tb: relDist}
	\end{table}
	
	\newpage
	\subsection{Effect of the kick direction}
	Lyapunov exponents characterize how chaotic an orbit of a black hole is. \autoref{fig: lyapunov17} shows how chaos affects triaxial galaxies, as a change of 1.9 \% significantly changes the orbits of kicked black holes. Examining images from \autoref{ch: Galaxies}, it has been found that there is a dependency with the magnitude of the velocity. This result, although unexpected, might have to do with the fact that small changes in the phase space of low energy orbits have higher impact that the same changes in high energy orbits, as the relative energy change is smaller, as can be seen in \autoref{fig: escapePhaseSpace} and \autoref{fig: escapeInner}.
	
	Furthermore, in oblate galaxies the behavior of the Lyapunov exponent is controlled by the $z$ component of the initial speed, as seen in figures \ref{fig: g2}, \ref{fig: g4}-\ref{fig: g9}, \ref{fig: g11}, \ref{fig: g14}, and \ref{fig: g17}. In this type of galaxies more chaotic orbits are seen in the $xy$-plane, where depending on the size of the $y$-semiaxis results are evenly distributed in this plane, while for $a_2 \leq a_1$ more chaotic orbits are seen in the $x$-axis (e.g. \autoref{fig: g20}). Both results seem to point that more chaotic orbits are related with the highest potential axis (mayor semiaxis), probably because induced torques in the orbits are higher when close to the $x$-axis of the phase space.
	\begin{figure}[h]
		\centering
		\includegraphics[width = 0.7\linewidth]{"../Files/Week 14/lyapunov_orbits"}
		\caption{Two closest orbits of the minimum Lyapunov exponent in galaxy \ref{fig: g17}, difference in the initial conditions if of 1.9 \%.}
		\label{fig: lyapunov17}
	\end{figure}