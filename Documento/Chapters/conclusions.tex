% Chapter Template

\chapter{Conclusions}\label{Conclusions}
In the present work, a study of the dynamics of the orbits of kicked black holes has been presented. Older studies in the literature considered only spherical distributions of masses and potentials to simulate the gravitational attraction and interaction of a kicked black hole with its host galaxy. However, through the document we have made a significant effort to show how triaxiality in galaxies can shift the predictions of the previous studies. Our simulations have found that spherical studies tend to under-estimate the return times and masses of a kicked black hole. In fact, on average, return times for triaxial studies are 2.6 times longer than those expected in spherical galaxies, masses, on the other hand, show an average increase of 24 \%, for black holes with the same seed mass ($10^5$ \sm) and same mass galaxies ($10^8$ \sm). Also, we have shown how the lower-estimation of the return parameters changes with the initial speed of a recoiling black hole. These results imply that triaxiality must be taken into account for statistical estimations of the black hole population in the universe. 

\vspace{1cm}

Furthermore, correlation of the return properties (time and mass) have been quantified at 0.63, and its dependance with the triaxiality of the galaxy and the initial speed of the lunches has been discussed. Distributions of the final descriptors of the simulated black holes have been studied, and the probability of finding a black hole such as the ULAS J1342+0928 quasar in the simulations has been calculated at 0.34 \%. With respect to quantifying the amount of chaos associated to each orbit, Lyapunov exponents have shown that more chaotic orbits are expected at the mayor semiaxis of the host galaxy.  

\vspace{1cm}

As for spherical galaxies, the effect of the baryonic fraction of the galaxy, the power law exponent of the gas profile, and the amount of stars have been discussed, all of which drastically alter the return properties of a black hole. Moreover, the study of the stellar fraction allowed to propose a fitting equation for the computational results, which show a divergent behavior with respect to the initial speed, and a parabola as a function of the stellar fraction.

\vspace{1cm}

Lastly, the stability of the Leapfrog integrating scheme has been tested, finding that although there are local total energy variations at different instants of a simulation, they are below 0.35 \%, and additionally, globally energy is conserved.