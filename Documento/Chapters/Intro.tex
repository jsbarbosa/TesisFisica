% !TeX spellcheck = en_US
% Chapter 1

%\chapter{Chapter Title Here} % Main chapter title
%
%\label{Chapter1} % For referencing the chapter elsewhere, use \ref{Chapter1} 

%----------------------------------------------------------------------------------------

% Define some commands to keep the formatting separated from the content 
%\newcommand{\code}[1]{\texttt{#1}}
%\newcommand{\file}[1]{\texttt{\bfseries#1}}
%\newcommand{\option}[1]{\texttt{\itshape#1}}

%----------------------------------------------------------------------------------------

\chapter{Introduction}
	The Theory of General Relativity by Albert Einstein was published in 1915, from which arise predictions such as gravitational waves, gravitational lenses, and time dilation. The term "gravitational waves" was introduced for the first time in a Henri Poincaré publication of 1905, in which he proposed the first equation for an invariant gravitational field before Lorentz transformations \cite{straumann2012general, bassan2014advanced}. At present, gravitational waves are understood as the periodic variations of the geometry of space-time, and have their origin in that the energy and moment density of a gravitational field act in turn as sources of gravity \cite{hoyng2006gravitational}. Although more than 100 years have passed since the publication of the theory, even today there are gaps in the understanding and implications of Einstein's equations. The foregoing is due, in part, to the difficulty of solving the equations for physical situations of interest. For example, gravitational waves can only be solved analytically for weak fields by using a linear form of these equations. However, at the experimental level it is only possible to detect gravitational waves from highly massive bodies, such as binary black hole systems, which only have strong fields. In 2016 a gravitational wave was detected for the first time in the history of mankind, and this discovery was recognized by the scientific community in 2017 with the Nobel Prize in Physics \cite{brugmann2018fundamentals}.
	
	In particular for binary systems (two bodies orbiting around their center of mass), there is a phenomenon known as \textit{recoil} or \textit{kick}. This is because, when considering General Relativity, the movement of bodies generates waves that carry momentum and energy, and that alters the trajectories previously predicted by the Theory of Universal Gravitation by Sir. Isaac Newton, where the solution to the equations of motion are elliptical trajectories in accordance with the laws of Kepler \cite{hughes2005black, hoyng2006gravitational, brugmann2018fundamentals}. This causes that the orbit little by little decay, and the two objects merge into a single body. At the moment the fusion takes place, the amplitude of the wave increases considerably. This implies that there will be a movement of the new body in the opposite direction of the propagation of the wave, given by the conservation of linear momentum. It is this movement that is known by the name of recoil or kick and was first described by Bonnor and Rotenberg in 1966 \cite{hughes2005black, bonnor1966gravitational}.
	
	The fusion of two black holes of a binary system gives rise to a new one, being this one of the mechanisms by which supermassive black holes are generated. This type of black hole has been found in almost all galaxies \cite{choksi2017recoiling}, and is characterized by having masses between $10^4$ \sm to $10^{10}$ \sm (\sm, solar masses), with such masses, it is possible for black holes to give rise to quasars. It has also been found that its mass correlates with properties of the galaxy among which are the speed of dispersion, luminosity and the mass of the galactic bulge. It has even been thought that these correlations show a co-evolution process of black holes and their galaxies. Among the effects of setbacks on black holes, it has been found that they limit the formation of black holes to masses less than $10^{10}$ \sm \cite{choksi2017recoiling}.
	
	Simulations of merger kicks have shown that the speed at which the resulting black hole moves, strongly depends on the relative orientation of the spins of the colliding black holes \cite{baker2008modeling}. \citeauthor{baker2008modeling} developed a set of equations fitting their numerical results, in which the mass ratio of the black holes ($q \equiv m_2 / m_1 \leq 1$), the spin vectors and the binary orbital angular momentum vector $\vec{a}_\text{1, 2} = \vec{S}_\text{1, 2}/m_\text{1, 2}$ are taken into account. By using the set of equations of \citeauthor{baker2008modeling}, \citeauthor{tanaka2009assembly} worked in the distribution of kick velocities of black holes with different masses and relative spin orientations. Their results show that speeds up to 3000 km/s are possible, slightly lower than the previous limit of 4000 km/s from older numerical relativistic studies.
	
	\begin{figure}[h]
		\centering
		\includegraphics[width=0.9\linewidth]{"../Files/Week 10/Tanaka"}
		\caption{Kick velocity distributions for different relative masses of the coalesced black holes. Random spin distributions are shown on the right side of the figure, while spins aligned with orbital momentum are shown on the right.}
		\label{fig: tanakaSpeeds}
	\end{figure}
	
	When a black hole experiences a kick, the gravitational force, the dynamic friction, and the accretion act on it. The dynamic friction is due to the interaction of a body in movement with non-collisional matter such as dark matter and stars, which generates drag on it decreasing its speed. On the other hand, given the gravitational field of the black hole, small bodies that are in its path will be incorporated increasing the mass of the same, which causes a decrease in the speed due to conservation of the momentum. Lastly, the cosmological acceleration is due to the expansion of the universe and has a constant value for a given redshift ($Z$) \cite{choksi2017recoiling}.
	
	Their effect on the black holes trajectory has been previously studied by \citeauthor{choksi2017recoiling}, finding that variations in the value of the cosmological acceleration have little effect on the simulations \cite{choksi2017recoiling}. For the accretion, they determined that the increase of this factor lowers the time it takes for a given black hole to return to its initial position, being this the most relevant effect for small-mass black holes. With respect to dynamic friction, they opted for a hybrid description between the models proposed by \citeauthor{ostriker1999dynamical} and \citeauthor{escala2005role}, managing to take into account both the subsonic range and the highly supersonic range of the drag force \cite{ostriker1999dynamical, escala2005role}. Finally, in their study they considered a spherically symmetric potential for the halo of the galaxy, bringing together the contributions of dark matter and visible matter in the same potential as mass distributions \cite{choksi2017recoiling}. Nevertheless, all of these works consider only spherical/symmetrical potentials for the host galaxy of the black hole.
	
	In this work we seek to analyze the effect of different triaxial potentials of the host galaxy, and initial velocities of the black hole in the return times and masses of the black hole to its galaxy. The above is of particular importance because with these potentials the angular momentum is not always conserved, the trajectories are not closed and the phase space is chaotic. The latter means that small variations in the initial conditions give rise to completely different final results. In addition, these potentials are observed in elliptical galaxies that rotate slowly \cite{buote2002chandra, binney1978elliptical}. Not to mention that the chaotic component represents a challenge for the integration methods of the motion equation, since a numerical error is not different from a change in an initial condition. Because of this the Leapfrog method is choosed to integrate the orbits, and its conservation of energy is studied.
	
	For this reason it is sought to perform each simulation using different numerical integrators, available in the Python and C library, REBOUND \cite{larson2017modeling}.
	
%\section{Objetivo general}
%	Estudiar el efecto de distintos potenciales triaxiales, velocidades iniciales e integradores num\'ericos sobre los tiempos requeridos por un agujero negro supermasivo para volver a su posici\'on inicial, luego de experimentar un retroceso, as\'i como cuantificar qué tan caótica es su trayectoria.
%	
%\section{Objetivos específicos}
%	\begin{itemize}
%		\item Obtener distribuciones de probabilidad para los tiempos de retorno en función de cada uno de los par\'ametros libres del potencial triaxial, la magnitud y direcci\'on de la velocidad inicial
%		\item Cuantificar qué tan caótica es la trayectoria del agujero negro en cada simulación, usando exponentes de Lyapunov
%		\item Evaluar el desempe\~no de los integradores n\'umericos usando la informaci\'on de las simulaciones
%	\end{itemize}