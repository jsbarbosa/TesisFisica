% !TeX spellcheck = en_US
% Chapter 1

%\chapter{Chapter Title Here} % Main chapter title
%
%\label{Chapter1} % For referencing the chapter elsewhere, use \ref{Chapter1} 

%----------------------------------------------------------------------------------------

% Define some commands to keep the formatting separated from the content
%\newcommand{\option}[1]{\texttt{\itshape#1}}

%----------------------------------------------------------------------------------------
\chapter{Spherical study}
			
	\section{Results}
	For a single simulation, the following data are saved: iteration time, current position, speed, and the black hole mass. With this information, accelerations and densities can be later reconstructed as on \autoref{fig: overallOutput}.
	\begin{figure}[h]
		\centering
		\includegraphics[width = 0.52\textwidth]{"../Files/Week 6/properties_s02v70"}
		\caption{Upper two plots show the output of a single simulation, while the lower one shows most the local properties per data point.}
		\label{fig: overallOutput}
	\end{figure}
	
	\subsection{Effect of the baryonic fraction}
	\begin{figure}[h]
		\centering
		\includegraphics[width = 0.7\textwidth]{"../Files/Week 5/baryonic_fraction_comparison"}
		\caption{.}
		\label{fig: baryonicfraction}
	\end{figure}
		
	\subsection{Effect of the power law exponent}
	\begin{figure}[h]
		\centering
		\begin{subfigure}[b]{0.49\textwidth}
			\includegraphics[width = \textwidth]{"../Files/Week 6/power_law"}
			\caption{.}
			\label{fig: powerLawOrbits}
		\end{subfigure}
		~ 
		\begin{subfigure}[b]{0.49\textwidth}
			\includegraphics[width=\textwidth]{"../Files/Week 6/power_law_density"}
			\caption{.}
			\label{fig: powerLawDensities}
		\end{subfigure}
		\caption{.}
		\label{fig: powerLaw}
	\end{figure}
	
	\subsection{Effect of the stellar fraction}
	\begin{figure}[h]
		\centering
		\includegraphics[width = 0.7\textwidth]{"../Files/Week 7/Symmetric/returntimes_stellar_speed"}
		\caption{.}
		\label{fig: stellarfraction}
	\end{figure}