% !TeX spellcheck = en_US
% Chapter Template

\chapter{Computational setup} % Main chapter title
	\section{Units}
	Computer simulations are sensitive to rounding errors due to the lack of infinite precision when representing decimal numbers. Really small numbers as well as really big ones tend to have bigger errors than those close to the unity, as can be seen on \autoref{fig: IEEE-754}.
	\begin{figure}[h]
		\centering
		\includegraphics[width=0.8\linewidth]{"../Files/Week 3/floating"}
		\caption{Floating point precision for different values, for a 32 bit and 64 bit holders.}
		\label{fig: IEEE-754}
	\end{figure}
	
	Under the International System of Units, distances are measured on meters, times on seconds, and masses on kilograms, nevertheless black holes are too heavy to be measured on kilograms, galaxies sizes too big to be quantified on meters, and time scales too large for a second. Because of that, the following units will be used throughout this document:
	\begin{table}[h]
		\centering
		\caption{Units of measure used on the simulations.}
		\label{tb: units}
		\begin{tabular}{c|c}
			\hline
			\textbf{Physical property} & \textbf{unit} \\
			\hline
			Length & 1 kilo-parsec (kpc) \\
			Mass & $10^5$ solar masses ($10^5$ \sm) \\
			Time & 1 giga-year (Gyr) \\
			\hline
		\end{tabular}
	\end{table}
	
	Along with the change of units, the universal gravitational constant and the Hubble parameter values are required to change.
	
	\subsection{Universal gravitational constant}
	First quantified by Henry Cavendish the gravitational constant has a value of $G_0 = 6.67408\times10^{-11}$ on SI units of m$^3$s$^{-2}$kg$^{-1}$. With the units of length, mass and time on \autoref{tb: units}, the constant of gravity used is given by:
	\begin{equation}
	\begin{array}{ccl}
	G & = & G_0 \left(\dfrac{1 \text{ kpc}^3}{\left(3.0857\times10^{19}\right)^3  \text{ m}^3}\right)\left(\dfrac{\left(3.154\times10^{16}\right)^2 \text{ s}^2}{1 \text{ Gyr}^2}\right)\left(\dfrac{1.98847\times10^{35} \text{ kg}}{10^5 M_\theta}\right) \\
	& = & 0.4493 \quad \dfrac{\text{kpc$^3$}}{\text{Gy$r^210^5$\sm}}	
	\end{array}
	\end{equation}
	
	
	\subsection{Hubble parameter}
	The Hubble constant is frequently used as $H_0 = 67.66 \pm 0.42$ kms$^{-1}$Mpc$^{-1}$ \cite{aghanim2018planck}, stating the speed of an astronomical body on kms$^{-1}$ at a distance of 1 Mpc. Nevertheless, the hubble constant has units of 1/time, thus, taking into account the units on \autoref{tb: units} one gets:
	\begin{equation}
	\begin{array}{ccl}
	H & = & H_0 \left(\dfrac{1 \text{ kpc}}{3.0857\times10^{16} \text{ km}}\right)\left(\dfrac{3.154\times10^{16} \text{ s}}{1 \text{ Gyr}}\right)\left(\dfrac{1 \text{ Mpc}}{1000 \text{ kpc}}\right) \\
	& \approx & 1.023 H_0 \times10^{-3} \text{ Gyr$^{-1}$} \\ 
	& = & 6.916\times10^{-2}\text{ Gyr$^{-1}$}
	\end{array}
	\end{equation}
	
	Although the Hubble parameter is often called Hubble constant, its value changes with time as can be seen on \autoref{fig: hubbleTime}. %In particular, at $z = 20$, the moment at which the kick occurs $H$ has a value of 3.699 Gyr$^{-1}$.
	\begin{figure}[h]
		\centering
		\includegraphics[width=0.8\linewidth]{"../Files/Week 5/hubble_time"}
		\caption{Dependency of the Hubble parameter with redshift.}
		\label{fig: hubbleTime}
	\end{figure}
	
	\section{Critical density and Virial Radius}\label{sec: cd_vr}
	Mass distributions used for the simulation of the host galaxy, are divergent for distances up to infinity. Because of this, the cumulative mass of all bodies within a given distance is called the virial mass and its value is taken as the mass of the whole system. The distance taken to calculate the virial mass is called virial radius ($R_\text{vir}$), and it is defined as the distance at which the average density of the galaxy is 200 times the critical density of the universe ($\rho_\text{crit}$).
	\begin{equation}\label{eq: critical_density}
	\rho_\text{crit} = \dfrac{3H(t)^2}{8\pi G}
	\end{equation}
	
	\begin{equation}\label{eq: R_vir_def}
	\begin{array}{c}
	\dfrac{M(R_\text{vir})}{V(R_\text{vir})} = \bar{\rho}(R_\text{vir}) =  200 \rho_\text{crit} = 75\dfrac{H(t)^2}{\pi G}\\
	\text{where $M(R_\text{vir})$ is the cumulative mass, and $V(R_\text{vir})$: the volume}
	\end{array}			
	\end{equation}
	
	The relation on \autoref{eq: critical_density} is found by considering the case where the geometry of the universe is flat, as a consequence it is said that the critical density is the minimum density required to stop the expansion of the universe \cite{binney2011galactic}.
	
\chapter{Time integration}
	Although \autoref{eq: equationMotion} is a one body equation of motion, it is a second order differential equation with no analytical solution due to the complexity of gravitational and dynamical friction components. Thus, to evolve the position of the black hole in time, numerical integration of the equation is carried on using the Leapfrog method.
	
	Graphically the differential equation is integrated in three steps, as seen on \autoref{fig: leapfrog}. First, from the acceleration at the current position $i$, a mid-point velocity is found ($\vec{v}_{i + 1/2}$). With this velocity, the position at the next time step is calculated. Finally, the velocity at $i + 1$ is found using the acceleration at this spot.	
	\begin{figure}[h]
		\centering
		\includegraphics[width = 0.8\linewidth]{"../Files/Week 13/leapfrog"}
		\caption{Leapfrog integration scheme}
		\label{fig: leapfrog}
	\end{figure}

	These steps are shown on \autoref{eq: leapfrog1} to \autoref{eq: leapfrog3}, where $a_i \equiv \ddot{\vec{x}}(\vec{x}_i, \dot{\vec{x}}_i)$ from \autoref{eq: equationMotion}, and $\dot{\vec{x}}_i \equiv \vec{v}_i$.
	
	\begin{equation}\label{eq: leapfrog1}
		\vec{v}_{i+1/2} = \vec{v}_i + \vec{a}_i\left(\dfrac{\Delta t}{2}\right)
	\end{equation}
	\begin{equation}
		\vec{x}_{i+1} = \vec{x}_i + \vec{v}_{i+1/2}\Delta t
	\end{equation}
	\begin{equation}\label{eq: leapfrog3}
		\vec{v}_{i+1} = \vec{v}_{i+1/2}+\vec{a}_{i+1}\left(\dfrac{\Delta t}{2}\right)
	\end{equation}
	
	Since the Leapfrog integration scheme, does not yield and analytical solution, it is sensitive to numerical errors. Because of this, an error assessment is done comparing the energy for nondissipative simulations, for both, the spherical and triaxial cases ($a_1 = a_2 = a_3 = 1$). By removing the second term on \autoref{eq: equationMotion}, stable orbits are found, and energy should be conserved. From the first law of thermodynamics, conservation of energy is given by:
	\begin{equation}
		E_T = K + V = \dfrac{1}{2}m_0v_0^2 + m_0\Phi^0_\text{grav} = \dfrac{1}{2}m_iv_i^2 + m_i\Phi^i_\text{grav}
	\end{equation}
	
	The gravitational potential is the sum of the potential generated by dark matter ($\Phi_\text{DM}$), stars ($\Phi_\text{stars}$) and gas ($\Phi_\text{gas}$) at a distance $r$.
	\begin{table}[h]
		\centering
		\caption{$\Phi$ values for the studied density profiles}
		\begin{tabular}{c|c}
			\hline
			\textbf{Profile} & $\Phi(r)$ \\
			\hline
			\rule{0pt}{4ex} 
			\textbf{NFW} & $-\dfrac{4\pi G\rho_{0}^\text{DM}R_s^3}{r}\ln\left(1 + \dfrac{r}{R_s}\right)$\\
			\textbf{Hernquist} & $-\dfrac{Gf_sf_bM_h}{r + \mathcal{R}_s}$ \\
			\textbf{Power-law} & $- \frac{4 \pi G \rho_0^\text{gas} \left(r + r_{0}\right)^{- n} \left(2 n r^{2} r_{0}^{n + 4} + r^{3} r_{0}^{n + 3} \left(n - 1\right) + r r_{0}^{n + 5} \left(n + 3\right) - 2 r_{0}^{6} \left(r + r_{0}\right)^{n} + 2 r_{0}^{n + 6}\right)}{r r_{0}^{3} \left(n - 3\right) \left(n - 2\right) \left(n - 1\right)}
			$
			\\
			\hline
		\end{tabular}
	\end{table}
	
	To check for energy changes over time, a total of 106 orbits were made, following \citeauthor{poon2001orbital}. These simulations lasted for almost half the age of the universe.
	\begin{figure}[h]
		\centering
		\includegraphics[width = 0.8\linewidth]{"../Files/Week 11/Comparison"}
		\caption{Energy variations of the Leapfrog scheme}
		\label{fig: energyAssestment}
	\end{figure}

	\autoref{fig: energyAssestment} shows a maximum fluctuation for the spherical case of 0.312 \% after more than a hundred dynamical times, while for the triaxial case oscillations are much smaller with a maximum amplitude of 0.001 \%. Despite the local changes in energy, energy is conserved globally using the Leapfrog scheme.

\chapter{Lyapunov exponents}
	In chaotic behavior, infinitesimally close initial conditions lead to evolutions that diverge exponentially fast. The Maximum Lyapunov Exponent $\mathcal{L}$, is an indicative of the rate of such divergence \cite{morbidelli2002modern}.
	\begin{figure}[h]
		\centering
		\includegraphics[width = 0.7\linewidth]{"../Files/Week 9/lyapunov_explain"}
		\caption{Representation of three arbitrary close orbits, and their evolution in time.}
		\label{fig: lyapunov_explain}
	\end{figure}
	
	Consider the upper two orbits ($\mathcal{O}^\text{ref}$, $\mathcal{O}^\text{var}$) in \autoref{fig: lyapunov_explain}, with initial conditions $\vec{x}^\text{ ref}(0)$, $\vec{p}^\text{ ref}(0)$ and $\vec{x}^\text{ var}(0)$, $\vec{p}^\text{ var}(0)$. Denoting the distance in each of the components of the phase space as:
	\begin{equation}\label{eq: deltax}
		\delta\vec{x}(t) = \vec{x}^\text{ ref}(t) - \vec{x}^\text{ var}(t) = \left(x^\text{ref}(t) - x^\text{var}(t), y^\text{ref}(t) - y^\text{var}(t), z^\text{ref}(t) - z^\text{var}(t)\right)
	\end{equation}
	\begin{equation}\label{eq: deltap}
		\delta\vec{p}(t) = \vec{p}^\text{ ref}(t) - \vec{p}^\text{ var}(t) = \left(p_x^\text{ref}(t) - p_x^\text{var}(t), p_y^\text{ref}(t) - p_y^\text{var}(t), p_z^\text{ref}(t) - p_z^\text{var}(t)\right)
	\end{equation}
	
	the Maximum Lyapunov Exponent can be written as \cite{morbidelli2002modern, munoz2015chaotic}:
	\begin{equation}
		\mathcal{L} = \lim_{t\rightarrow\infty}\dfrac{1}{t}\ln\dfrac{\left|\delta\vec{x}(t), \delta\vec{p}(t)\right|}{\left|\delta\vec{x}(0), \delta\vec{p}(0)\right|}
	\end{equation}
	
	where $|\delta\vec{x}(t), \delta\vec{p}(t)|$ is the Euclidean norm of the 6 dimensional phase space. The numerical calculation of $\mathcal{L}$ requires special care, as a computation up to infinity must be done \cite{morbidelli2002modern}. In 1980 a technique by \citeauthor{benettin1980lyapunov} solved this problem, as The Maximum Lyapunov Exponent can be calculated as follows:
	\begin{enumerate}
		\item Define an arbitrary initial distance in the phase space $\delta\vec{x}(0) = (\delta x_0, \delta y_0, \delta z_0)$, $\delta\vec{p}(0) \equiv 0$.
		\item Simulate both $\mathcal{O}^\text{ref}$ and $\mathcal{O}^\text{var}$ until a predefined time $T$.
		\item Calculate the distance in phase space at time $T$ between the reference orbit and the variational one (equations \ref{eq: deltax} and \ref{eq: deltap}).
		\item Calculate the coefficient $s_i$.
		\begin{equation}
			s_i = \dfrac{\left|\delta\vec{x}_i(T), \delta\vec{p}_i(T)\right|}{\left|\delta_i\vec{x}(0), \delta\vec{p}_i(0)\right|}
		\end{equation}
		\item For the new iteration, $\delta_{i + 1}\vec{x}(0) = \delta_i\vec{x}(T) / s_i$, and $\delta_{i + 1}\vec{p}(0) = \delta_i\vec{p}(T) / s_i$
		\item Repeat $l$ times, to obtain:
		\begin{equation}
			\mathcal{L} = \dfrac{\sum_{i = 1}^l \ln(s_i)}{lT}
		\end{equation}
	\end{enumerate} 
	