% !TeX spellcheck = en_US
% Chapter 1

%\chapter{Chapter Title Here} % Main chapter title
%
%\label{Chapter1} % For referencing the chapter elsewhere, use \ref{Chapter1} 

%----------------------------------------------------------------------------------------

% Define some commands to keep the formatting separated from the content
%\newcommand{\option}[1]{\texttt{\itshape#1}}

%----------------------------------------------------------------------------------------
\chapter{Triaxial study}
	\section{Setup}	
	The host galaxy is modeled as a dark matter halo, stars and gas, just as the spherical case. Much of the profiles for each of the components remains the same, the only difference is that a thin shell of uniform density will have the geometry of an ellipsoid, and not that of a sphere. This is achieved by defining an ellipsoid radius $m$, that can be replaced for the spherical radius $r$, on equations \ref{eq: dmdensity}, \ref{eq: sdensity} and \autoref{eq: rdensity}. Using the Cartesian coordinates $x_1$, $x_2$, $x_3$ and $a_1$, $a_2$, $a_3$ the semi-axis of the ellipsoid, $m$ is defined as follows:
	\begin{equation}
		m^2(\vec{x}) \equiv a_1^2\left[\left(\dfrac{x_1}{a_1}\right)^2 + \left(\dfrac{x_2}{a_2}\right)^2 + \left(\dfrac{x_3}{a_3}\right)^2\right] = x_1^2 + \left(\dfrac{a_1}{a_2}\right)^2x_2^2 + \left(\dfrac{a_1}{a_3}\right)^2x_3^2
	\end{equation}
	
	A thin shell, whose inner and outer skins are the surfaces $m$ and $m + \delta m$ is described by \autoref{eq: m2}, where $\tau \geq 0$ labels the surfaces \cite{binney2011galactic}.
	\begin{equation}\label{eq: m2}
		m^2(\vec{x}, \tau) = a_1^2\left(\frac{x_1^{2}}{\tau + a_{1}^{2}} + \frac{x_2^{2}}{\tau + a_{2}^{2}} + \frac{x_3^{2}}{\tau + a_{3}^{2}}\right)
	\end{equation}
	
	Densities are used for the calculation of the dynamical friction and accretion onto the black hole. Although one might think that by integrating the density over an elliptical volume, the acceleration due to gravity would be given by $a_\text{grav} = GM(m)/m^2$, the later is not true because two points $(x_1, x_2, x_3)$ and $(x_1', x_2', x_3')$ might have the same cumulative mass at $m$ (black line), but the effective gravitational mass acting at each point is completely different (blue and orange lines) as it is shown in \autoref{fig: triaxial_mass_issue}.
	
	\begin{figure}[h]
		\centering
		\includegraphics[width = 0.5\linewidth]{"../Files/Week 7/triaxial_mass_issue"}
		\caption{Although the cumulative mass at the orange and blue dots is the same, the effective gravitational mass is different.}
		\label{fig: triaxial_mass_issue}
	\end{figure}
	
	Because of this, the potential due to a given triaxial density must be found. Calculating the gravitational potential for such configuration, challenged some great minds of the XVIII and XIX centuries \cite{binney2011galactic}. To do so, the contributions of all ellipsoidal shells that make up the profile are taken into account, following \citeauthor{binney2011galactic}:
	\begin{equation}
		\psi(m) \equiv \int\limits_{0}^{m^2} \rho(m^2)dm^2 = \int\limits_{0}^{k = m^2} \rho(k)dk \qquad \text{defining $k \equiv m^2$} 
	\end{equation}
	
	The potential of any body in which $\rho = \rho(m^2)$ is \cite{binney2011galactic}:
	\begin{equation}\label{eq: generalPotential}
		\Phi(\vec{x}) = -\pi G \dfrac{a_2a_3}{a_1}\int\limits_{0}^{\infty}\dfrac{\psi(\infty) - \psi(m)}{\sqrt{(\tau + a_1^2)(\tau + a_2^2)(\tau + a_3^2)}}d\tau \qquad m = m(\vec{x}, \tau)
	\end{equation}
	
	Most of the triaxials potentials cannot be analytically integrated, nevertheless it can be done numerically if the integral is not improper and converges. To make the integral proper, the following change of variable is done:
	\begin{equation}
		\omega = \dfrac{\tau^\gamma}{\tau^\gamma + 1}, \qquad \tau = \left(\frac{\omega}{1-\omega}\right)^{\frac{1}{\gamma}}, \qquad d\tau = \dfrac{\left(- \frac{\omega}{\omega - 1}\right)^{\frac{1}{\gamma}}}{\gamma \omega \left(- \omega + 1\right)}
	\end{equation} 
	
	Since the gravitational acceleration is given by the gradient of the potential, to numerically calculate the gradient, a total of 6 numerical integrals must be done (two for each dimension). Another option is to take advantage of the fact that $\vec{x}$ and $\tau$ are independent variables, thus:
	\begin{equation}
		\nabla \int f(\vec{x}, \tau)d\tau = \int [\nabla f(\vec{x}, \tau)] d\tau
	\end{equation}
	
	By doing this, the number of numerical integrals reduces to 3. Defining a vector $\vec{\phi}$, whose components are given by:
	\begin{equation}
		\phi_i(x_i, \tau) = \dfrac{x_i}{\left(\tau + a_i^2\right)^{\frac{3}{2}} \prod\limits_{i \neq j}^3\sqrt{\tau + a_j^2}}, \qquad \vec{\phi}(\vec{x}, \tau) = (\phi_1(x_1, \tau), \phi_2(x_2, \tau), \phi_3(x_3, \tau))
	\end{equation}
	
	Potentials for each of the components of the galaxy are found by finding $\psi(\infty) - \psi(m)$ and replacing on \autoref{eq: generalPotential}.
	
	\subsection{Dark matter halo}
	\begin{figure}[h]
		\centering
		\begin{subfigure}[b]{0.49\textwidth}
			\includegraphics[width = \textwidth]{"../Files/Week 7/symmetric"}
			\caption{Spherical case}
			\label{fig: symmetricDensity3d}
		\end{subfigure}
		~ 
		\begin{subfigure}[b]{0.49\textwidth}
			\includegraphics[width=\textwidth]{"../Files/Week 7/triaxial"}
			\caption{Triaxial case with ($a_1$:$a_2$:$a_3$) = (1:0.5:0.3)}
			\label{fig: triaxialDensity3d}
		\end{subfigure}
		\caption{Dark matter densities comparison.}
		\label{fig: symmetricTriaxial}
	\end{figure}
	\begin{equation}
		\begin{array}{rl}
			\nabla \Phi_\text{DM}(\vec{x}) & = 
			\displaystyle\int\limits_{0}^{\infty}
			\dfrac{2 \pi G R_{s}^{3}\rho_0 a_{1} a_{2} a_{3}}{m(\vec{x}, \tau)\left(R_{s} + m(\vec{x}, \tau)\right)^{2}}
			\vec{\phi}(\vec{x}, \tau) d\tau \\
			& = 2 \pi G R_{s}^{3}\rho_0 a_{1} a_{2} a_{3} \displaystyle\int\limits_{0}^{\infty}
			\dfrac{\vec{\phi}(\vec{x}, \tau) d\tau}{m(\vec{x}, \tau)\left(R_{s} + m(\vec{x}, \tau)\right)^{2}}	
		\end{array}
	\end{equation}
	
	\subsection{Stellar profile}
	\begin{equation}
		\begin{array}{rl}
			\nabla \Phi_\text{S}(\vec{x}) & = \displaystyle\int\limits_{0}^{\infty} \frac{G M_{s} a_{1} a_{2} a_{3}}{m(\vec{x}, \tau)\left(\mathcal{R_s} + m(\vec{x}, \tau)\right)^{3}}
			\vec{\phi}(\vec{x}, \tau) d\tau \\
			& = G M_{s} a_{1} a_{2} a_{3} \displaystyle\int\limits_{0}^{\infty} \frac{ \vec{\phi}(\vec{x}, \tau) d\tau}{m(\vec{x}, \tau)\left(\mathcal{R_s} + m(\vec{x}, \tau)\right)^{3}}
		\end{array}
	\end{equation}
	\subsection{Gas profile}
	\begin{equation}
		\nabla \Phi_\text{G}(\vec{x}) = 2 \pi G \rho_0 a_{1} a_{2} a_{3}
		\left\{
		\begin{array}{l}
		\displaystyle\int\limits_{0}^{\infty} \vec{\phi}(\vec{x}, \tau) d\tau \qquad \text{for $m(\vec{x}, \tau) < r_0$}\\
		r_{0}^{- n} \displaystyle\int\limits_{0}^{\infty} m(\vec{x}, \tau)^{n}  \vec{\phi}(\vec{x}, \tau) d\tau \qquad \text{else}
		\end{array}
		\right.		
	\end{equation}
	\section{Results}
	\subsection{Numerical errors}
	\begin{figure}[h]
		\centering
		\begin{subfigure}[b]{0.49\textwidth}
			\includegraphics[width = \textwidth]{"../Files/Week 7/error"}
			\caption{Errors on potential}
			\label{fig: potentialErrors}
		\end{subfigure}
		~ 
		\begin{subfigure}[b]{0.49\textwidth}
			\includegraphics[width=\textwidth]{"../Files/Week 7/symmetric_triaxial"}
			\caption{Cumulative errors on simulations}
			\label{fig: simulationErrors}
		\end{subfigure}
		\caption{Differences for analytical and numerical integration of the potentials. Analytical is taken as: $-GM(r) / r^2$}
		\label{fig: numericalErrors}
	\end{figure}

	\begin{figure}[h]
		\centering
		\begin{subfigure}[b]{0.49\textwidth}
			\includegraphics[width = \textwidth]{"../Files/Week 7/orthogonal_triaxial"}
			\caption{Orbits for three orthogonal launches}
			\label{fig: orthogonalLaunches}
		\end{subfigure}
		~ 
		\begin{subfigure}[b]{0.49\textwidth}
			\includegraphics[width=\textwidth]{"../Files/Week 7/ellipsoid"}
			\caption{Elliptical geometry}
		\end{subfigure}
		\caption{Orthogonal launches for a triaxial profile with semi-axis ($a_1$:$a_2$:$a_3$) = (1:0.99:0.95)}
		\label{fig: mainOrthogonalLaunches}
	\end{figure}