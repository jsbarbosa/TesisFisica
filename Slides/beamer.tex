% !TeX spellcheck = en_US
\documentclass{beamer}
%\documentclass[handout]{beamer}
\usepackage[english]{babel}
\usepackage[utf8]{inputenc}
\usepackage{graphicx}
\usepackage{amsmath}
\usepackage{tikz}
\usepackage{multirow}
\usepackage[normalem]{ulem}
\usepackage{subcaption}
%\usepackage{tcolorbox}
\usefonttheme{structuresmallcapsserif}
\usetheme{Antibes}

\usecolortheme{whale}
\setbeamertemplate{footline}[frame number]

\usepackage[backend = bibtex, style = verbose, sorting = none, autocite = footnote]{biblatex}
\addbibresource{../Documento/bibliography.bib}

%\usepackage[table, dvipsnames]{xcolor}

\newcommand{\sm}[0]{$M_\odot$}
\newcommand{\erf}[1]{\text{erf}\left(#1\right)}
\def\checkmark{{\color{red}\tikz\fill[scale=0.4](0,.35) -- (.25,0) -- (1,.7) -- (.25,.15) -- cycle;}}

\newcommand\blfootnote[1]
{%
	\begingroup
	\renewcommand\thefootnote{}\footnote{#1}%
	\addtocounter{footnote}{-1}%
	\endgroup
}
\newcommand{\fcite}[1]{\blfootnote{\tiny\cite{#1}}}

\expandafter\def\expandafter\insertshorttitle\expandafter{%
	\insertshorttitle\hfill%
	\insertframenumber\,/\,\inserttotalframenumber}

\begin{document}
\begin{frame}
	\centering
%	\color{white}
	\textsc{\LARGE Orbits of black holes in triaxial potentials}
	\\
	\vspace{2.5cm}
	Juan Barbosa\\
	\vspace{1cm}
	\small
	Jaime Forero\\ 
	Advisor\\
	\vspace{0.5cm}
	\footnotesize
	Departamento de F\'isica\\
	Facultad de Ciencias\\
	Universidad de los Andes
\end{frame}

%\begin{frame}{Overview}
%	\tableofcontents
%\end{frame}

%\section{Introduction}
%\begin{frame}{Introduction}
%	\begin{columns}
%		\begin{column}{0.5\textwidth}
%			\begin{figure}[h]
%				\centering
%				\includegraphics[width=0.9\linewidth]{images/Albert_Einstein_Head}
%			\end{figure}
%		\end{column}
%		\begin{column}{0.5\textwidth}
%			\begin{itemize}
%				\item Theory of General Relativity, 1916
%				\item Although more than 100 years have passed since the publication of the theory, even today there are gaps in the understanding and implications of Einstein's equations
%			\end{itemize}
%		\end{column}
%	\end{columns}
%\end{frame}

%\begin{frame}{Introduction}
%	\begin{figure}
%		\centering
%		\includegraphics[width=0.4\linewidth]{images/example}
%		\caption{\href{run:/home/juan/Documents/TesisFisica/Slides/images/super_kick.mp4}{Binary black hole explorer}}
%	\end{figure}
%	
%	\fcite{varma2018binary}
%\end{frame}

%\begin{frame}{Objectives}
%	\small
%	Study the effect of different triaxial potentials, and initial speeds \sout{and numerical integrators} on the times required by a supermassive black hole to return to its initial position, after experiencing a recoil, as well as to quantify how chaotic its trajectory is.
%	
%	\begin{itemize}
%		\item Obtain probability distributions for the return times based on each of the free parameters of the triaxial potential, the magnitude and direction of the initial velocity
%		\item Quantify how chaotic is the trajectory of the black hole in each simulation, using exponents of Lyapunov
%		\item Evaluate the performance of the \sout{numerical integrators} LeapFrog scheme, using the information of the simulations
%	\end{itemize}
%\end{frame}

%\section{Methodology}
%\subsection{Galatic setup}
%\begin{frame}{Galactic setup}
%	\begin{figure}[h]
%		\centering
%		\includegraphics[width=0.8\linewidth]{../Documento/Figures/NGC4414_modified}
%		\caption{NGC4414 galaxy as seen by the Hubble telescope.}
%	\end{figure}
%\end{frame}
%
%\begin{frame}
%	\begin{enumerate}
%		\item Gas density (Power law):
%		\begin{equation}
%			\rho_\text{gas}(r) = \left \{
%			\begin{matrix}
%			\rho_0^\text{gas} & \text{if $r < r_0$}\\
%			\rho_0^\text{gas}\left(\dfrac{r_0}{r}\right)^{-n} & \text{if $r \geq r_0$}\\
%			\end{matrix}
%			\right.
%		\end{equation}
%		\item Gas density (Double power law)
%		\begin{equation}\label{eq: rdensity}
%		\rho_\text{gas}(r) = \dfrac{\rho_0^\text{gas}}{\left(1 + \dfrac{r}{r_0}\right)^n}
%		\end{equation}
%	\end{enumerate}
%\end{frame}
%
%\begin{frame}
%	\begin{enumerate}
%		\begin{columns}
%			\begin{column}{0.45\textwidth}
%				\item Gas density (Power law)
%				\begin{figure}[h]
%					\centering
%					\includegraphics[width=\linewidth]{images/power_law_density_old}
%				\end{figure}
%			\end{column}
%		
%			\begin{column}{0.45\textwidth}
%				\item Gas density (Double power law)
%				\begin{figure}[h]
%					\centering
%					\includegraphics[width=\linewidth]{"../Files/Week 6/power_law_density"}
%				\end{figure}
%			\end{column}
%		\end{columns}
%	\end{enumerate}
%\end{frame}

%\begin{frame}
%	\begin{enumerate}
%		\item Dark matter (NWF):
%		\begin{equation}\label{eq: dmdensity}
%		\rho_\text{DM}(r) = \dfrac{\rho_0^\text{DM}}{\frac{r}{R_s}\left(1 + \frac{r}{R_s}\right)^2}
%		\end{equation}
%		\item Stellar density (Hernquist):
%		\begin{equation}
%		\rho_s(r) = \dfrac{f_sf_bM_h \mathcal{R}_s}{2\pi r(r + \mathcal{R}_s)^3}
%		\end{equation}
%		\item Gas density (Power law):
%		\begin{equation}
%		\rho_\text{gas}(r) = \left \{
%		\begin{matrix}
%		\rho_0^\text{gas} & \text{if $r < r_0$}\\
%		\rho_0^\text{gas}\left(\dfrac{r_0}{r}\right)^{-n} & \text{if $r \geq r_0$}\\
%		\end{matrix}
%		\right.
%		\end{equation}
%	\end{enumerate}
%\end{frame}

%\begin{frame}
%	Mass distributions of the host galaxy, are divergent. The end of the host is taken at the virial radius.
%	\begin{equation}\label{eq: critical_density}
%	\rho_\text{crit} = \dfrac{3H(t)^2}{8\pi G}
%	\end{equation}
%	
%	\begin{equation}\label{eq: R_vir_def}
%	\dfrac{M(R_\text{vir})}{V(R_\text{vir})} = \bar{\rho}(R_\text{vir}) =  200 \rho_\text{crit} = 75\dfrac{H(t)^2}{\pi G}
%	\end{equation}
%	
%	where $M(R_\text{vir})$ is the cumulative mass, and $V(R_\text{vir})$: the volume
%\end{frame}

%\begin{frame}
%	Mass distributions between components are given by:
%	\begin{equation}
%		M_\text{DM}(R_\text{vir}) = (1 - f_b)M_h
%	\end{equation}
%	\begin{equation}
%		M_\text{stars}(R_\text{vir}) = f_sf_bM_h
%	\end{equation}
%	\begin{equation}
%		M_\text{gas}(R_\text{vir}) = (1 - f_s)f_bM_h
%	\end{equation}
%	
%	with $f_b = 0.156$ and $M_h = 10^8$ \sm
%	\begin{equation}
%		R_\text{vir} = \left({\dfrac{M_hG}{100 H(t)^2}}\right)^{1/3}
%	\end{equation}
%\end{frame}

%\subsection{Units}
%\begin{frame}{}
%	\begin{table}[h]
%		\centering
%		\caption{Units of measure used on the simulations.}
%		\label{tb: units}
%		\begin{tabular}{c|c}
%			\hline
%			\textbf{Physical property} & \textbf{unit} \\
%			\hline
%			Length & 1 kilo-parsec (kpc) \\
%			Mass & $10^5$ solar masses ($10^5$ \sm) \\
%			Time & 1 giga-year (Gyr) \\
%			\hline
%		\end{tabular}
%	\end{table}
%	\begin{enumerate}
%		\item Universal gravitational constant:
%		\begin{equation}
%			G = 0.4493 \quad \dfrac{\text{kpc$^3$}}{\text{Gy$r^210^5$\sm}}
%		\end{equation}
%		\item Hubble parameter:
%			\begin{equation}
%			H \approx 1.023 H_0 \times10^{-3} \text{ Gyr$^{-1}$} = 6.916\times10^{-2}\text{ Gyr$^{-1}$}
%			\end{equation}
%	\end{enumerate}
%\end{frame}


%\subsection{Equation of motion}
%\begin{frame}{Equation of motion}
%	Trajectories of the kicked black holes are obtained by numerically solving the equation of motion.
%	\begin{equation}\label{eq: equationMotion}
%		\ddot{\vec{x}} = -a_\text{grav}(\vec{x})\hat{x} + \left(a_\text{DF}(\vec{x}, \dot{\vec{x}})-\dot{x}\dfrac{\dot{M_\bullet}(x, \dot{x})}{M_\bullet}\right)\dot{\hat{x}} 
%	\end{equation}
%	
%	where $M_\bullet$ is the black hole mass
%	
%	\begin{itemize}
%		\item Dark matter, stars and gaseous materials from the medium interact with the black hole adding a drag force.
%		\item The black hole accretes matter from the surroundings.
%	\end{itemize}
%\end{frame}

\section{Studies}
\subsection{Symmetrical}

\subsubsection{Results}
%\begin{frame}{Effect of the stellar fraction}
%	\begin{figure}[h]
%		\centering
%		\includegraphics[width=\linewidth]{"../Files/Week 9/PhaseSpace_escape"}
%	\end{figure}
%\end{frame}
%
%\begin{frame}{Effect of the stellar fraction}
%	\begin{figure}[h]
%		\centering
%		\includegraphics[width=\linewidth]{"../Files/Week 9/PhaseSpace_in"}
%		%		\caption{Return time for different stellar densities and speeds.}
%	\end{figure}
%\end{frame}

%\begin{frame}{Results}
%	\begin{columns}
%		\begin{column}{0.3\linewidth}
%			\begin{figure}[h]
%				\centering
%				\includegraphics[height=0.7\textheight]{"../Files/Week 6/properties_s02v70"}
%			\end{figure}
%		\end{column}
%		\begin{column}{0.7\linewidth}
%			\begin{figure}[h]
%				\centering
%				\includegraphics[width = \linewidth]{"../Files/Week 6/properties_s02v70_1"}
%				\caption{Distances and speeds.}
%			\end{figure}
%		\end{column}
%	\end{columns}
%\end{frame}
%
%\begin{frame}{Results}
%	\begin{columns}
%		\begin{column}{0.3\linewidth}
%			\begin{figure}[h]
%				\centering
%				\includegraphics[height=0.7\textheight]{"../Files/Week 6/properties_s02v70"}
%			\end{figure}
%		\end{column}
%		\begin{column}{0.7\linewidth}
%			\begin{figure}[h]
%				\centering
%				\includegraphics[width = \linewidth]{"../Files/Week 6/properties_s02v70_2"}
%				\caption{Mass of the black hole.}
%			\end{figure}
%		\end{column}
%	\end{columns}
%\end{frame}
%
%\begin{frame}{Results}
%	\begin{columns}
%		\begin{column}{0.3\linewidth}
%			\begin{figure}[h]
%				\centering
%				\includegraphics[height=0.7\textheight]{"../Files/Week 6/properties_s02v70"}
%			\end{figure}
%		\end{column}
%		\begin{column}{0.7\linewidth}
%			\begin{figure}[h]
%				\centering
%				\includegraphics[width = \linewidth]{"../Files/Week 6/properties_s02v70_3"}
%				\caption{Multiple properties of a single simulation.}
%			\end{figure}
%		\end{column}
%	\end{columns}
%\end{frame}
%
%\begin{frame}{Effect of the baryonic fraction}
%	\begin{figure}[h]
%		\centering
%		\includegraphics[height=0.7\textheight]{"../Files/Week 5/baryonic_fraction_comparison"}
%		\caption{5 \% increase on baryonic fraction decreases return times.}
%	\end{figure}
%\end{frame}
%
%\begin{frame}{Effect of the power law exponent}
%	\begin{figure}[h]
%		\centering
%		\begin{tabular}{cc}
%			\includegraphics[width = 0.5\textwidth]{"../Files/Week 6/power_law"} & \includegraphics[width = 0.5\textwidth]{"../Files/Week 6/power_law_density"}
%		\end{tabular}
%		\caption{Smaller exponents increase the return time.}
%	\end{figure}
%\end{frame}

\begin{frame}{Effect of the stellar fraction}
	\begin{figure}[h]
		\centering
		\includegraphics[width=0.9\linewidth]{"../Files/Week 10/returntimes_stellar_speed"}
		\caption{Return time for different stellar densities.}
	\end{figure}
\end{frame}

\begin{frame}{Effect of the stellar fraction}
	\begin{figure}[h]
		\centering
		\includegraphics[width=\linewidth]{"../Files/Week 10/returntimes_speed"}
		\caption{Return time for different initial speeds.}
	\end{figure}
\end{frame}

\begin{frame}{Effect of the stellar fraction}
	\small
	\begin{equation}\label{eq: fitTr}
	\log_{10}(T_\text{return}) = [a(f_s) v + b(f_s)] + \dfrac{c(f_s)}{v - 1.3}
	\end{equation}
	\begin{table}[h]
		\centering
		\begin{tabular}{ccc}
			\includegraphics[width = 0.33\textwidth]{"../Files/Week 10/a"} & 
			\includegraphics[width = 0.33\textwidth]{"../Files/Week 10/b"} & 
			\includegraphics[width = 0.33\textwidth]{"../Files/Week 10/c"}
		\end{tabular}
	\end{table}
	\small
	\begin{equation}
		a(f_s) = 232f_s^2 + 25 f_s + 2.83
	\end{equation} 
	\begin{equation}
		b(f_s) = -40.7 f_s - 0.75
	\end{equation}
	\begin{equation}
		c(f_s) = 60 f_s^2 - 2.8 f_s - 0.080
	\end{equation}
\end{frame}

\begin{frame}{Effect of the stellar fraction}
	\begin{figure}[h]
		\centering
		\includegraphics[width=0.8\linewidth]{"../Files/Week 10/surface"}
		\caption{Return time for different stellar densities and speeds.}
	\end{figure}
\end{frame}

\subsection{Triaxial}
\begin{frame}{Initial conditions}
	\begin{figure}[h]
		\centering
		\begin{subfigure}[b]{0.475\textwidth}
			\includegraphics[width = \textwidth]{"../Files/Week 13/3d_initial_speeds"}
			\caption{Cartesian}
		\end{subfigure}
		~ 
		\begin{subfigure}[b]{0.475\textwidth}
			\includegraphics[width=\textwidth]{"../Files/Week 13/polar_initial_speeds"}
			\caption{Polar}
		\end{subfigure}
		\caption{Distributions of initial speeds for the triaxial lunches. $\theta$ describes the polar angle and $\phi$ the azimuth.}
		\label{fig: initialSpeedDistributions}
	\end{figure}
\end{frame}

\begin{frame}{Initial conditions}
	\begin{figure}[h]
		\centering
		\includegraphics[width = 0.9\linewidth]{"../Files/Week 13/triaxial_axes"}
		\caption{Distribution of the 30 pair of values for the $y$ and $z$ semiaxis.}
		\label{fig: semiaxisDist}
	\end{figure}
\end{frame}

\begin{frame}{Results}
	\begin{columns}
		\begin{column}{0.4\textwidth}
			\begin{figure}[h]
				\centering
				\includegraphics[width=\linewidth]{"../Files/Week 14/dist_masses"}
				\caption{Mass distributions of the returned black hole, for the 30 triaxial lunches (blue) and for an spherical galaxy.}
				\label{fig: massDist}
			\end{figure}
		\end{column}
		\begin{column}{0.6\textwidth}
			\begin{figure}[h]
				\centering
				\includegraphics[width = 0.8\linewidth]{"../Files/Week 14/dist_times"}
				\caption{Return time distributions, for the 30 triaxial lunches (blue) and for an spherical galaxy. Below, the cumulative probability of the purple curve.}
				\label{fig: timeDist}
			\end{figure}
		\end{column}
	\end{columns}
\end{frame}

\begin{frame}{Results}
	\begin{figure}[h]
		\centering
		\includegraphics[width = 0.8\linewidth]{"../Files/Week 14/redshift_dist"}
		\caption{Comparison of the predicted distribution of quasars and the observational data.}
		\label{fig: universeDist}
	\end{figure}
\end{frame}

\begin{frame}{Results}
	\begin{figure}[h]
		\centering
		\begin{subfigure}[t]{0.35\textwidth}
			\includegraphics[width = \textwidth]{"../Files/Week 13/images/10_time"}
			\caption{Return times}
		\end{subfigure}
		~ 
		\begin{subfigure}[t]{0.35\textwidth}
			\includegraphics[width=\textwidth]{"../Files/Week 13/images/10_mass"}
			\caption{Return masses}
		\end{subfigure}
		\begin{subfigure}[t]{0.35\textwidth}
			\includegraphics[width=\textwidth]{"../Files/Week 13/images/10_lyapunov"}
			\caption{Lyapunov exponent}
		\end{subfigure}
		\begin{subfigure}[t]{0.35\textwidth}
			\includegraphics[width=\textwidth]{"../Files/Week 13/images/10_ellipsoid"}
			\caption{Geometry}
		\end{subfigure}
		\caption{Distribution of the different properties for the galaxy with $a_1 = 1$, $a_2 = 9.6\times10^{-1}$, $a_3 = 7.0\times10^{-1}$.}
	\end{figure}
\end{frame}

\begin{frame}{Results}
	\begin{figure}[h]
		\centering
		\begin{subfigure}[t]{0.35\textwidth}
			\includegraphics[width = \textwidth]{"../Files/Week 13/images/3_time"}
			\caption{Return times}
		\end{subfigure}
		~ 
		\begin{subfigure}[t]{0.35\textwidth}
			\includegraphics[width=\textwidth]{"../Files/Week 13/images/3_mass"}
			\caption{Return masses}
		\end{subfigure}
		\begin{subfigure}[t]{0.35\textwidth}
			\includegraphics[width=\textwidth]{"../Files/Week 13/images/3_lyapunov"}
			\caption{Lyapunov exponent}
		\end{subfigure}
		\begin{subfigure}[t]{0.35\textwidth}
			\includegraphics[width=\textwidth]{"../Files/Week 13/images/3_ellipsoid"}
			\caption{Geometry}
		\end{subfigure}
		\caption{Distribution of the different properties for the galaxy with $a_1 = 1$, $a_2 = 6.9\times10^{-1}$, $a_3 = 1.2\times10^{-2}$.}
	\end{figure}
\end{frame}


\begin{frame}{Results}
	\begin{figure}[h]
		\centering
		\begin{subfigure}[t]{0.35\textwidth}
			\includegraphics[width = \textwidth]{"../Files/Week 13/images/24_time"}
			\caption{Return times}
		\end{subfigure}
		~ 
		\begin{subfigure}[t]{0.35\textwidth}
			\includegraphics[width=\textwidth]{"../Files/Week 13/images/24_mass"}
			\caption{Return masses}
		\end{subfigure}
		\begin{subfigure}[t]{0.35\textwidth}
			\includegraphics[width=\textwidth]{"../Files/Week 13/images/24_lyapunov"}
			\caption{Lyapunov exponent}
		\end{subfigure}
		\begin{subfigure}[t]{0.35\textwidth}
			\includegraphics[width=\textwidth]{"../Files/Week 13/images/24_ellipsoid"}
			\caption{Geometry}
		\end{subfigure}
		\caption{Distribution of the different properties for the galaxy with $a_1 = 1$, $a_2 = 6.6\times10^{-2}$, $a_3 = 6.1\times10^{-2}$.}
	\end{figure}
\end{frame}
%\begin{frame}{Triaxial setup}
%	Density shells for each profile are ellipsoids
%	\begin{equation}
%	m^2(\vec{x}) \equiv x_1^2 + \left(\dfrac{a_1}{a_2}\right)^2x_2^2 + \left(\dfrac{a_1}{a_3}\right)^2x_3^2
%	\end{equation}
%	
%	A thin shell, whose inner and outer skins are the surfaces $m$ and $m + \delta m$ is described by:
%	\begin{equation}\label{eq: m2}
%	m^2(\vec{x}, \tau) = a_1^2\left(\frac{x_1^{2}}{\tau + a_{1}^{2}} + \frac{x_2^{2}}{\tau + a_{2}^{2}} + \frac{x_3^{2}}{\tau + a_{3}^{2}}\right)
%	\end{equation}
%	where $\tau \geq 0$ labels the surfaces \fcite{binney2011galactic}
%\end{frame}
%
%
%\begin{frame}
%	\begin{columns}
%		\begin{column}{0.3\linewidth}
%			\begin{figure}[h]
%				\centering
%				\includegraphics[width = \linewidth]{"../Files/Week 7/triaxial_mass_issue"}
%				\caption{Effective gravitational mass}
%				\label{fig: triaxial_mass_issue}
%			\end{figure}
%		\end{column}
%		\begin{column}{0.7\linewidth}
%			The contributions of all ellipsoidal shells that make up the profile are taken into account
%			\begin{equation}
%				\psi(m) \equiv \int\limits_{0}^{m^2} \rho(m^2)dm^2 = \int\limits_{0}^{k = m^2} \rho(k)dk
%			\end{equation}
%			
%			The potential of any body in which $\rho = \rho(m^2)$ is:
%		\end{column}
%	\end{columns}
%	\begin{equation}\label{eq: generalPotential}
%	\Phi(\vec{x}) = -\pi G \dfrac{a_2a_3}{a_1}\int\limits_{0}^{\infty}\dfrac{\psi(\infty) - \psi(m)}{\sqrt{(\tau + a_1^2)(\tau + a_2^2)(\tau + a_3^2)}}d\tau
%	\end{equation}
%	\fcite{binney2011galactic}
%\end{frame}
%
%\begin{frame}
%	Numerical integration of the gradients of the potentials is made with Simpson 3/8 rule.
%	\begin{equation}
%		\nabla \Phi_\text{DM}(\vec{x}) = 2 \pi G R_{s}^{3}\rho_0 a_{1} a_{2} a_{3} \displaystyle\int\limits_{0}^{\infty}
%		\dfrac{\vec{\phi}(\vec{x}, \tau) d\tau}{m(\vec{x}, \tau)\left(R_{s} + m(\vec{x}, \tau)\right)^{2}}	
%	\end{equation}
%	\begin{equation}
%		\nabla \Phi_\text{S}(\vec{x})= G M_{s} a_{1} a_{2} a_{3} \displaystyle\int\limits_{0}^{\infty} \frac{ \vec{\phi}(\vec{x}, \tau) d\tau}{m(\vec{x}, \tau)\left(\mathcal{R_s} + m(\vec{x}, \tau)\right)^{3}}
%	\end{equation}
%	
%	\begin{equation}
%		\small
%		\nabla \Phi_\text{G}(\vec{x}) = 2 \pi G \rho_0 a_{1} a_{2} a_{3}
%		\left\{
%		\begin{array}{l}
%		\displaystyle\int\limits_{0}^{\infty} \vec{\phi}(\vec{x}, \tau) d\tau \qquad \text{for $m(\vec{x}, \tau) < r_0$}\\
%		r_{0}^{- n} \displaystyle\int\limits_{0}^{\infty} m(\vec{x}, \tau)^{n}  \vec{\phi}(\vec{x}, \tau) d\tau \qquad \text{else}
%		\end{array}
%		\right.		
%	\end{equation}
%\end{frame}

%\subsubsection{Results}
%\begin{frame}{Triaxial}
%	\begin{figure}[h]
%		\centering
%		\begin{subfigure}[t]{0.45\textwidth}
%			\includegraphics[width = \textwidth]{"../Files/Week 7/symmetric"}
%%			\caption{Spherical case}
%%			\label{fig: symmetricDensity3d}
%		\end{subfigure}
%		~ 
%		\begin{subfigure}[t]{0.45\textwidth}
%			\includegraphics[width=\textwidth]{"../Files/Week 7/triaxial"}
%%			\caption{Triaxial case with ($a_1$:$a_2$:$a_3$) = (1:0.5:0.3)}
%%			\label{fig: triaxialDensity3d}
%		\end{subfigure}
%		\begin{subfigure}[t]{0.6\textwidth}
%			\includegraphics[width=\textwidth]{"../Files/Week 7/ellipsoid_"}
%%			\caption{Equicentred ellipsoids with ($a_1$:$a_2$:$a_3$) = (1:0.5:0.3)}
%		\end{subfigure}
%		\caption{Dark matter densities comparison between spherical and triaxial cases.}
%		\label{fig: symmetricTriaxial}
%	\end{figure}
%\end{frame}

%\begin{frame}
%	\begin{figure}[h]
%		\begin{tabular}{cc}
%			\includegraphics[width = 0.7\textwidth]{"../Files/Week 10/3dorbit"} &
%			\includegraphics[width = 0.3\textwidth]{"../Files/Week 10/3dorbit_distances"}
%		\end{tabular}
%	\end{figure}
%\end{frame}

%
%\begin{frame}{Results}
%	\begin{figure}[h]
%		\centering
%		\begin{tabular}{cc}
%			\includegraphics[width = 0.49\textwidth]{"../Files/Week 7/error"} & 
%			\includegraphics[width = 0.49\textwidth]{"../Files/Week 7/symmetric_triaxial"}
%		\end{tabular}
%		\caption{Differences for analytical and numerical integration of the potentials. Analytical is taken as: $GM(r) / r^2$}
%		\label{fig: numericalErrors}
%	\end{figure}
%\end{frame}

%\begin{frame}
%	\begin{figure}[h]
%		\begin{tabular}{cc}
%			\includegraphics[width = 0.6\textwidth]{"../Files/Week 7/orthogonal_triaxial"} &
%			\includegraphics[width = 0.4\textwidth]{"../Files/Week 7/ellipsoid"}
%		\end{tabular}
%		\caption{Orthogonal launches for a triaxial profile with semi-axis ($a_1$:$a_2$:$a_3$) = (1 : 0.99 : 0.95)}
%		\label{fig: mainOrthogonalLaunches}
%	\end{figure}
%\end{frame}
%
%\begin{frame}{}
%	\begin{figure}[h]
%		\begin{tabular}{cc}
%			\includegraphics[width = 0.6\textwidth]{"../Files/Week 11/lyapunov_distances"} &
%			\includegraphics[width = 0.4\textwidth]{"../Files/Week 11/lyapunov_orbits"}
%		\end{tabular}
%	\end{figure}
%\end{frame}

%\begin{frame}
%	\begin{center}
%		\scshape\huge
%		Thank You
%	\end{center}
%\end{frame}

%
%
%
%
%
%
%
%
%
%
%
%
%
%
%
%
%
%
%
%
%\begin{frame}
%	Computer simulations are sensitive to rounding errors due to the lack of infinite precision when representing decimal numbers.
%	\begin{figure}[h]
%		\centering
%		\includegraphics[width=0.7\linewidth]{"../Files/Week 3/floating"}
%		\caption{Floating point precision for different values, for a 32 bit and 64 bit holders.}
%		\label{fig: IEEE-754}
%	\end{figure}
%\end{frame}
%
%\begin{frame}
%	Mass distributions between components are given by:
%	\begin{equation}
%		M_\text{DM}(R_\text{vir}) = (1 - f_b)M_h
%	\end{equation}
%	\begin{equation}
%		M_\text{stars}(R_\text{vir}) = f_sf_bM_h
%	\end{equation}
%	\begin{equation}
%		M_\text{gas}(R_\text{vir}) = (1 - f_s)f_bM_h
%	\end{equation}
%	
%	with $f_b = 0.156$ and $M_h = 10^8$ \sm
%	\begin{equation}
%		R_\text{vir} = \left({\dfrac{M_hG}{100 H(t)^2}}\right)^{1/3}
%	\end{equation}
%\end{frame}
%
%\begin{frame}{Dynamical friction}
%	As the black hole travels through the galaxy, dark matter, stars and gaseous materials from the medium interact with the black hole adding a drag force.
%	\begin{figure}[h]
%		\centering
%		\includegraphics[width = 0.25\linewidth]{images/dyn_friction}
%		\caption{Collisionless dynamical friction}
%	\end{figure}
%	\fcite{df_image}
%\end{frame}
%
%\begin{frame}
%	\begin{itemize}
%	\item Collisionless matter interacts with the black hole gravitational only
%	\begin{equation}\label{eq: df_cl}
%	a_\text{DF}^\text{cl}(\vec{x}, \dot{\vec{x}}) = -\dfrac{4\pi G^2}{\dot{x}^2} M_\bullet\rho(\vec{x})\ln\Lambda\left(\erf{X} - \dfrac{2}{\sqrt{\pi}}Xe^{-X^2}\right)
%	\end{equation}
%	
%	\item On the other side gas is in direct contact with the black hole
%	\begin{equation}\label{eq: df_c}
%	a^\text{c}_\text{DF}(\vec{x}, \dot{\vec{x}}) = -\dfrac{4\pi G^2}{\dot{x}^2}M_\bullet\rho_\text{gas}(\vec{x})f(\mathcal{M})
%	\end{equation}
%	\end{itemize}
%\end{frame}
%
%\begin{frame}{Accretion into the black hole}
%	As the black hole accretes matter from the surroundings, an acceleration appears, due to the second law of Newton.
%	\begin{equation}
%	\dot{M}_\bullet(\vec{x}, \dot{\vec{x}}) = \left\{
%	\begin{array}{lc}
%	\dot{M}_\bullet^\text{BHL}(\vec{x}, \dot{\vec{x}}) & \text{if $\dot{M}_\bullet^\text{BHL} < \dot{M}_\bullet^\text{Edd}$} \\
%	\dot{M}_\bullet^\text{Edd} & \text{else}
%	\end{array}
%	\right.
%	\end{equation}
%	
%	\begin{equation}
%	\dot{M}_\bullet^\text{BHL}(\vec{x}, \dot{\vec{x}}) = \dfrac{4\pi G^2 \rho_G(\vec{x})M^2_\bullet}{\left(c_s^2 + \dot{x}^2\right)^{3/2}} % \qquad \text{with } \rho_B(\vec{x}) = \rho_\text{stars}(\vec{x}) + \rho_\text{gas}(\vec{x})
%	\end{equation}
%	\begin{equation}
%	\dot{M}_\bullet^\text{Edd} = \dfrac{(1 - \epsilon)M_\bullet}{\epsilon t_\text{Edd}} \qquad \epsilon = 0.1, \quad t_\text{Edd} = 0.44 \text{ Gyr}
%	\end{equation}
%\end{frame}
%
%\begin{frame}
%	Drag generated by gas depends on local sound speed
%	\begin{equation}
%		\mathcal{M}(\dot{x}) \equiv \dfrac{|\dot{x}|}{c_s} = |\dot{x}|\sqrt{\dfrac{\mathcal{M}_w}{\gamma RT_\text{vir}}} = |\dot{x}|\sqrt{\dfrac{\mathcal{M}_w}{\gamma R}\left(\dfrac{2k_BR_\text{vir}}{\mu m_p G M_h}\right)}
%	\end{equation}
%	\begin{equation*}
%		\mathcal{M}(\dot{x}) = 1.63|\dot{x}|\sqrt{\dfrac{R_\text{vir}}{M_h}}
%	\end{equation*}
%	
%	\begin{equation}\label{eq: df_c}
%	a^\text{c}_\text{DF}(\vec{x}, \dot{\vec{x}}) = -\dfrac{4\pi G^2}{\dot{x}^2}M_\bullet\rho_\text{gas}(\vec{x})f(\mathcal{M})
%	\end{equation}
%	
%	with
%	\begin{equation}\footnotesize
%	f(\mathcal{M}) = \left\{
%	\begin{matrix}
%	0.5\ln\Lambda \left[\erf{\dfrac{\mathcal{M}}{\sqrt{2}}} - \sqrt{\dfrac{2}{\pi}}\mathcal{M}e^{-\mathcal{M}^2/2}\right]& \text{if $\mathcal{M} \leq 0.8$}\\
%	1.5\ln\Lambda \left[\erf{\dfrac{\mathcal{M}}{\sqrt{2}}} - \sqrt{\dfrac{2}{\pi}}\mathcal{M}e^{-\mathcal{M}^2/2}\right] & \text{if $0.8 < \mathcal{M} \leq \mathcal{M}_{eq}$}\\
%	0.5\ln\left(1 - \mathcal{M}^{-2}\right) + \ln\Lambda & \text{if $\mathcal{M} > \mathcal{M}_{eq}$}
%	\end{matrix}
%	\right.
%	\end{equation}
%\end{frame}
%
%\begin{frame}
%	\begin{equation}
%		\omega = \dfrac{\tau^\gamma}{\tau^\gamma + 1}, \qquad \tau = \left(\frac{\omega}{1-\omega}\right)^{\frac{1}{\gamma}}, \qquad d\tau = \dfrac{\left(- \frac{\omega}{\omega - 1}\right)^{\frac{1}{\gamma}}}{\gamma \omega \left(- \omega + 1\right)}
%	\end{equation}
%	
%	\begin{equation}
%		\phi_i(x_i, \tau) = \dfrac{x_i}{\left(\tau + a_i^2\right)^{\frac{3}{2}} \prod\limits_{i \neq j}^3\sqrt{\tau + a_j^2}}
%	\end{equation}
%	\begin{equation}
%		\vec{\phi}(\vec{x}, \tau) = (\phi_1(x_1, \tau), \phi_2(x_2, \tau), \phi_3(x_3, \tau))
%	\end{equation}
%\end{frame}

\end{document}